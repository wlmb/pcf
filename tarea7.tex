\documentclass{exam}
\usepackage{espanhol}
\usepackage{bm}
\usepackage{amsmath}
\usepackage{comment}
%\usepackage{showkeys}
% \usepackage{pstricks}
\decimalpoint
\pointname{\ puntos}
\cfoot[]{\thepage}
\begin{document}
\begin{center}
  \bf\large Tarea 7\\
  Propedéutico Electrodinámica\\
  2020-05-03\\[20pt]
\end{center}
\hbox to \textwidth{Nombre: \enspace\hrulefill}
Entregar el viernes 2020-05-08.

\begin{questions}
\question\label{1} Encuentra la forma diferencial de la ley de inducción de
Faraday.

Pistas
\begin{itemize}
\item Escribe $\mathcal E=-(1/c)(d/dt)\Phi_B$ en términos de
  integrales de $\bm E$ y $\bm B$.
\item Usa el teorema de Stokes para llevarla a la forma de una
  integral de superficie cuyo valor es nulo.
\item Argumenta que esto sucede para cualquier superficie orientable,
  y por tanto, demuestra que el integrando es cero.
\end{itemize}
\question\label{2} Considera un alambre recto que alimenta mediante una
  corriente $I$ a un capacitor de placas paralelas formadas por discos
  metálicos de radio $a$ separados una distancia $d$. Demuestra que de
  acuerdo a la Ley de Ampère la circulación de $\bm B$ alrededor de
  una trayectoria cerrada que rodea al alambre es
  \begin{parts}
  \part $\frac{4\pi}{c}I$ si interpretamos a $\mathcal C$ como la
    frontera de una superficie $\mathcal S_1$ que cruza al alambre.
  \part 0 si interpretamos a la misma trayectoria $\mathcal C$ como la
    frontera de una superficie $\mathcal S_2$ que cruza por en medio de las placas
    del capacitor.
  \end{parts}
  Notas:
  \begin{itemize}
  \item En realidad, la circulación de $\bm B$ no debería depender de
    la elección de la superficie $\mathcal S_n$, sólo de la frontera
    $\mathcal C$.
  \item El problema se origina en que este sistema es {\em no
      estacionario} pues hay acumulación de carga.
  \item La Ley de Ampére no puede ser correcta para sistemas no estacionarios.
  \end{itemize}

\question\label{3} Considera una esfera de radio $a(t)$ cargada uniformemente en
  todo su interior con una carga $Q$ y que se expande con velocidad
  $v=da/dt=$cte. Calcula $\nabla\times B(\bm r,t)$ en un punto $r<a(t)$
  dentro de la esfera de acuerdo a la ley de Ampère.

  Pistas
  \begin{itemize}
  \item Calcula la densidad de corriente eléctrica en el interior de
    la esfera.
  \end{itemize}

  Notas:
  \begin{itemize}
  \item De acuerdo a la ley de Ampère, en todo circuito sobre la
    superficie de una esfera con $r<a$ hay circulación magnética.
  \item Eso implica que habría un campo $\bm B$ con componentes
    tangenciales.
  \item Pero el problema es isotrópico,
  \item y es imposible {\em peinar} a una esfera.
  \item Por tanto, la Ley de Ampère debe estar mal.
  \item De nuevo, este es un sistema no estacionario.
  \end{itemize}

\question\label{4} Usando la ecuación de continuidad demuestra que la ley de
  Ampère no puede ser correcta en sistemas no estacionarios.

  Pistas:
  \begin{itemize}
  \item La versión diferencial de la Ley de Ampére es $\nabla\times\bm
    B=(4\pi/c)\bm j$.
  \item Toma la divergencia de ambos lados.
  \item Usa la ecuación de continuidad para llegar a
    $0=\nabla\cdot\nabla\times\bm B=(4\pi/c)\nabla\cdot\bm
    j=-(4\pi/c)(\partial/\partial t)\rho$.
  \item Esta ecuación se puede violar en sistemas no estacionarios.
  \end{itemize}

\question\label{5} Generaliza la Ley de Ampère para que valga en sistemas no
  estacionarios.

  Pistas:
  \begin{itemize}
  \item Parte de la Ley de Gauss $\nabla\cdot\bm E=4\pi\rho$.
  \item Toma su derivada, $\nabla\cdot(\partial/\partial t)\bm
    E=4\pi(\partial/\partial t)\rho$.
  \item Usa la ley de continuidad para escribir
    $\nabla\cdot\left((\partial/\partial t)\bm E+4\pi\bm j\right)=0$.
  \item Argumenta que el término entre paréntesis se puede escribir
    como rotacional de {\em algo}.
  \item Usa la Ley de Ampère para identificar dicho algo en el caso
    estacionario es $c\bm B$.
  \item Generaliza y concluye con la {\em Ley de Ampère-Maxwell}
    $$
    \nabla\times\bm B=\frac{1}{c}\frac{\partial}{\partial t}\bm
    E+\frac{4\pi}{c}\bm j.
    $$
  \item El primer término de del lado derecho se conoce como {\em
      corriente de desplazamiento}.
  \end{itemize}
\question\label{6} Demuestra que la ley
  de Ampère-Maxwell resuelve los problemas encontrados en los
  problemas \ref{2}-\ref{4}.

  Pistas:
  \begin{itemize}
  \item Demuestra que la integral de la densidad de corriente en una
    sección del alambre del problema \ref{2} es igual a la integral de
    la corriente de desplazamiento entre las placas de un capacitor.
  \item Demuestra que la corriente de desplazamiento es igual y
    opuesta a la densidad de corriente eléctrica en el problema
    \ref{3}.
  \item El \ref{4} es trivial.
  \end{itemize}
\question\label{7} Demuestra que en el vacío el campo electromagnético cumple
  con la ecuación de onda.

  Pistas:
  \begin{itemize}
  \item Escribe las dos ecuaciones para los rotacionales de los campos
    en ausencia de cargas y corrientes.
  \item Toma el rotacional de una de éstas y sustituye la otra.
  \item Usa la identidad
    $\nabla\times\nabla\times(\ldots)=\nabla\nabla\cdot(\ldots)-\nabla^2(\ldots)$.
  \item Sustituye las ecuaciones para las divergencias.
  \end{itemize}


\question \label{21} Demuestra que la ecuación de onda tiene
  soluciones de la forma $\bm E(\bm r,t)=\bm E_0 F(x-ct)$ donde $F$ es
  una función arbitraria de un solo argumento.

  Pista:
  \begin{itemize}
  \item Sustituye en la ec. de onda.
  \end{itemize}
  Notas:
  \begin{itemize}
  \item Este tipo de solución se conoce como {\em ondas planas}, pues el
    campo es constante en los planos $x=$cte. Hay otros tipos de soluciones.
  \item La onda se propaga sin deformarse en la dirección $x$ con
    velocidad $c$.
  \item Podemos rotar la solución a cualquier otra dirección, generalizando la
    solución a $\bm E(\bm r,t)=\bm E_0 F(\hat {\bm n}\cdot\bm r-ct)$.
  \item Esta representa una onda plana que se mueve a velocidad $c$ en
    la dirección del vector unitario $\hat{\bm n}$.
  \item Los planos de campo constante son ortogonales a $\hat{\bm n}$,
    de la forma $\hat{\bm n}\cdot\bm r=$cte.
  \item El principio de superposición nos permite sumar ondas viajando
    en distintas direcciones.
  \item Si la onda depende sinusoidalmente del tiempo y de la
    posición, como en $\bm E(\bm r,t)=\bm E_0 \cos(\bm k\cdot\bm
    r-\omega t)$ se le conoce como onda plana {\em monocromática}.
  \item $\omega$ es la frecuencia, $\bm k=k\hat{\bm n}$ es el vector
    de onda y $k=\omega/c$ el número de onda. El periodo es
    $T=2\pi/\omega$ y la longitud de onda es $\lambda=2\pi/k$.
  \end{itemize}

\question \label{22} Demuestra que las ecuaciones de Maxwell en el
  vacío tienen solución en forma de ondas planas
  monocromáticas. Encuentra las relaciones entre el campo magnético y
  el campo eléctrico y la dirección de propagación.

  Pistas:
  \begin{itemize}
  \item Conviene emplear una notación compleja y proponer ondas del
    tipo $\bm E(\bm r,t)=\text{Re}\, \bm E_0 e^{i(\bm k.\cdot\bm
      r-\omega t)}$ y $\bm B(\bm r,t)=\text{Re}\, \bm B_0 e^{i(\bm k.\cdot\bm
      r-\omega t)}$.
  \item Sustituyendo en las ecuaciones de Maxwell para los
    rotacionales obtenemos las siguientes ecuaciones algebráicas
    $\bm k\times \bm E=(\omega/c)\bm B$ y
    $\bm k\times \bm B=-(\omega/c)\bm E$.
  \item Luego, $\bm k$, $\bm E$ y $\bm B$ son mutuamente
    perpendiculares y forman una triada ordenada derecha.
  \item Tomando el producto de una de las ecuaciones previas por $\bm
    k$ y sustituyendo en la otra, y empleando la identidad del álgebra
    vectorial $\bm k\times\bm k\times(\ldots)=\bm k\bm
    k\cdot(\ldots)-k^2(\ldots)$ y usando el inciso previo, obtenemos la
    relación de dispersión $k^2=(\omega/c)^2$.
  \item Por tanto $|E|=|B|$.
  \end{itemize}

 Notas:
 \begin{itemize}
 \item La ecuación de onda se deriva de las ecuaciones de Maxwell en
   el vacío, toda solución de las ecuaciones de Maxwell en el vacío
   satisfacen la ecuación de onda, pero no toda solución de la
   ecuación de onda satisface las ecuaciones de Maxwell.
 \end{itemize}

\question\label{23} Considera una onda electromagnética, plana,
  monocromática, propagándose
  en la dirección $z$, $\bm E(\bm r,t)=\text{Re}\,\bm E_0 e^{i(kz-\omega t)}$.
  \begin{parts}
  \part Demuestra que $\bm E_0=E_0\hat{\bm e}$ donde el vector de
    {\em polarización} unitario $\hat{\bm e}$ yace en el plano $xy$.
  \part Describe el movimiento de $\bm E(\bm 0,t)$ conforme transcurre
    el tiempo si $\hat{\bm e}=(1,0,0)$,
  \part $\hat{\bm e}=(\cos\theta,\sin\theta,0)$ para un ángulo
    $\theta$,
  \part $\hat{\bm e}=(1,i,0)/\sqrt2$,
  \part $\hat{\bm e}=(1,-i,0)/\sqrt2$.
  \part $\hat{\bm e}=(a,b i,0)/\sqrt{a^2+b^2}$,
  \end{parts}

  Pistas:
  \begin{itemize}
  \item Expresa $e^{-i\omega t}=\cos\omega t-i\sin\omega t$, haz los
    productos y toma la parte real del resultado.
  \end{itemize}
  Notas:
  \begin{itemize}
  \item La polarización puede ser lineal en cualquier dirección,
    circular con cualquier helicidad o elíptica con cualquier
    dirección de giro, excentricidad y semieje mayor.
  \end{itemize}


\question\label{8} Demuestra que los campos electromagnéticos se pueden
  escribir en términos de un potencial escalar y un potencial
  vectorial como
  $$
  \bm B=\nabla\times \bm A,
  $$
  $$
  \bm E=-\nabla\phi-\frac{1}{c}\frac{\partial}{\partial t}\bm A.
  $$

  Pistas:
  \begin{itemize}
  \item Resuelve la ley de Gauss magnética escribiendo a $\bm B$ como
    un rotacional.
  \item Sustituyelo en la ley de inducción de Faraday.
  \item Demuestra que $\nabla\times\left((\bm E+(1/c)(\partial/\partial
    t)\bm A\right)=0$,
  \item y por tanto el término entre paréntesis se puede escribir como
    un gradiente
  \end{itemize}

  Notas:
  \begin{itemize}
  \item Estos potenciales no necesariamente coinciden con el potencial
    Coulombiano que apareció en electrostática ni con el potencial
    vectorial que apareció en magnetostática.
  \item $\nabla\cdot\bm A$ es arbitraria.
  \end{itemize}


\question \label{9} Demuestra que un {\em cambio de norma} de la forma
  $$
  \bm A\to \bm \tilde{\bm A}=\bm A+\nabla\Lambda,
  $$
  $$
  \phi \to \tilde\phi=\phi-\frac{1}{c}\frac{\partial}{\partial t}\Lambda,
  $$
  con $\Lambda(\bm r,t)$ un campo escalar arbitrario, deja invariante
  al campo electromagnético: $\bm E=\tilde{\bm E}$ y $\bm
  B=\tilde{\bm E}$.

\question\label{10} Encuentra la ecuación que cumplen los potenciales $\phi$ y
  $\bm A$.


  Pistas:
  \begin{itemize}
  \item Sustituye los campos del problema \ref{8} en las ecuaciones de
    Maxwell inhomogéneas (ley de Gauss y ley de Ampère-Maxwell).
  \end{itemize}
\question \label{11}  Demuestra que $\phi$ y $\bm A$ cumplen ecuaciones de onda
  con fuentes $\rho$ y $\bm j/c$ respectivamente, en la {\em norma de
    Lorentz}
  $$\nabla\cdot\bm A+\frac{1}{c}\frac{\partial}{\partial t}\phi=0.$$

  Pistas:
  \begin{itemize}
  \item Sustituye la norma en las ecuaciones del problema \ref{10}.
  \end{itemize}


\question\label{12} Calcula el potencial vectorial $\bm A(\bm r)$ en una región
  permeada por un campo magnético uniforme y constante $\bm
  B=B\hat{\bm z}$ en la {\em norma de Landau} $A_x=0$.


\question\label{13} Demuestra que la circulación del potencial
  vectorial a lo largo de un circuito es igual al flujo magnético
  encerrado,
  $$
  \int_{\partial\mathcal S} d\bm l\cdot\bm A=\int_{\mathcal S}d\bm
  a\cdot\bm B,
  $$
  para toda superficie orientable $\mathcal S$.

  Pistas:
  \begin{itemize}
  \item Usa el teorema de Stokes.
  \end{itemize}
\question\label{14} Considera una bobina recta cilíndrica de longitud
  $\ell$ y radio $a\ll \ell$ con $N\gg 1$ espiras uniformemente
  enrolladas, en la que circula una corriente $I$. Calcula el
  potencial vectorial $\bm A$ que produce en todo el espacio.

  Pistas:
  \begin{itemize}
  \item El campo $\bm B$ ya fue calculado en un problema anterior.
  \item Utiliza el problema \ref{13}.
  \end{itemize}


\question\label{15} Demuestra que el campo electromagnético cumple la
  ley (local) de conservación de la energía.

  Pistas:
  \begin{itemize}
  \item La forma genérica de las leyes locales de conservación (todas
    las leyes de conservación, de acuerdo a la teoría de la
    relatividad) es
    $$
    \frac{d}{d t}(\text{cantidad})_V+(\text{flujo})_{\partial V}=(\text{fuente})_V
    $$
    donde
    $$
    (\text{cantidad})_V=\int_V d^3r\, \text{densidad},
    $$
    $$
    (\text{flujo})_{\partial V}=\int_{\partial V} d\bm
    a\cdot\text{densidad de flujo},
    $$
    $$
    (\text{fuente})_V=\int_V d^3r\, \text{densidad de fuentes},
    $$
    $V$ es un volumen cualquiera y $\partial V$ su frontera.
  \item Usando el teorema de Gauss, se puede escribir como
    $$\frac{\partial}{\partial
      t}\text{densidad}+\nabla\cdot\text{densidad de
      flujo}=\text{densidad de fuentes}.
    $$
  \item Para la energía, la densidad de fuentes (la energía
    que la materia da al campo por unidad de tiempo y por unidad de
y    volumen) es (-) el trabajo que el campo ejerce sobre la materia
    por unidad de tiempo por unidad de volumen.
  \item La densidad de energía se suele denotar por la letra $u$
  \item y la densidad de flujo de energía por la letra $\bm S$,
    conocida como el {\em vector de Poynting}.
  \item La fuerza por unidad de volumen es $f=\rho\bm E+(1/c)\bm
    j\times\bm B$.
  \item Luego, la potencia por unidad de volumen es $\bm E\cdot\bm j$.
  \item El ejercicio consiste entonces en hallar expresiones para $\bm S$ y $u$
    en términos del campo electromagnético $\bm E$ y $\bm B$ y que
    cumplan
    $$
    \frac{\partial}{\partial t}u+\nabla\cdot\bm S=-\bm E\cdot\bm j.
    $$
  \item Puedes empezar por el lado derecho, sustituir $\bm j$ en
    términos de $\bm B$ y $\bm E$ usando la ley de Ampère-Maxwell y
    llevar el resultado a la forma de arriba empleando identidades del
    cálculo vectorial.
  \item La densidad de energía resultante es
    $$
    u=\frac{E^2+B^2}{8\pi}.
    $$
  \item El vector de Poynting resultante es
    $$
    \bm S=\frac{c}{4\pi}\bm E\times \bm B.
    $$
  \end{itemize}
  Notas:
  \begin{itemize}
  \item Nota que el resultado es consistente con la expresión
    tentativa hallada en tareas previas para la energía electrostática
    y magnetostática.
  \item La ley de conservación de la energía electromagnética se
    conoce como {\em teorema de Poynting}.
  \end{itemize}


\question\label{16} Un resistor con resistencia $R$ consta de un
  cilindro de longitud $\ell$ y radio $a$ relleno de un material con
  conductividad $\sigma$. En el resistor fluye una corriente
  $I$. Calcula el flujo de energía electromagnética alrededor del
  resistor y compárala con la energía disipada en su interior.

  Pistas:
  \begin{itemize}
  \item A través del resistor hay una caida de potencial $V$.
  \item Luego, en su superficie hay un campo eléctrico $\bm E$
    paralelo al eje del resistor.
  \item Alrededor de la corriente $I$ circula un campo magnético $\bm
    B$.
  \item Calcula $\bm S$ e intégralo sobre las paredes del resistor.
  \end{itemize}
  Notas:
  \begin{itemize}
  \item La energía que disipa el resistor es proporcionada por el
    campo electromagnético,
  \item pero no llega al resistor a través de los conectores, sino
    viajando por el espacio vacío en dirección perpendicular a la corriente, a través de
    las paredes.
  \end{itemize}

\question\label{17} Un capacitor está formado por dos discos metálicos
  de radio $a$ separados una distancia $d\ll a$. Se conectan ambos
  discos por un resistor de resistencia $R$ en forma de alambre recto
  que conecta los centros de ambos discos. Al tiempo $t=0$ se colocan
  cargas $\pm Q_0$ en los discos y se deja que el sistema se descargue
  a través del resistor.  Verifica la forma integral de ley de
  conservación de energía en todo cilindro de radio $r<a$ coaxial con
  el capacitor.

  Pistas:
  \begin{itemize}
  \item Calcula la energía almacenada en un cilindro coaxial de radio
    $r$ integrando la densidad de energía.
  \item Calcula la potencia disipada por el resistor en el eje.
  \item Calcula el cambio en la energía almacenada conforme transcurre
    el tiempo.
  \item Calcula el flujo de energía en las paredes del cilindro.
  \item Verifica el balance entre los términos anteriores.
  \end{itemize}

\question\label{18} Demuestra que el campo electromagnético cumple con
  la Ley de Conservación del Ímpetu.

  Pistas:
  \begin{itemize}
  \item Análogamente al teorema de Poynting, la ley de conservación
    del ímpetu se puede escribir como
    $$
    \frac{\partial}{\partial t} \bm g +\nabla\cdot(-\bm T)=-\bm f,
    $$
    donde $\bm g$ es la densidad de ímpetu electromagnético, $-\bm T$
    es el flujo de ímpetu electromagnético y $-\bm f$ es la fuente de
    ímpetu para el campo, por unidad de tiempo y de volumen.
  \item $\bm f$ es entonces el ímpetu que el campo le proporciona a la
    materia por unidad de tiempo por unidad de volumen, es decir, la
    densidad de fuerza $\bm f=\rho\bm E+(1/c)\bm j\times\bm B$.
  \item Como el ímpetu es un vector, su flujo tiene dos direcciones
    (la dirección del ímpetu y la dirección en que fluye). Más
    precisamente, $-\bm T$ es un tensor de rango dos (una matriz)
    cuyas componentes son $-T_{ij}=$flujo de la componente $i$ del
    ímpetu que atraviesa una superficie normal a la dirección $j$ por
    unidad de área y unidad de tiempo.
  \item En una situación estacionaria, $\bm T$ nos da la fuerza
    electromagnética sobre un sistema,
    $$
    \bm F_V=\int_v d^3r\, f=\int da_j T_{ij}.
    $$
  \item $\bm T$ se conoce como el {\em tensor de esfuerzos}
    electromagnético.
  \item Se puede empezar eliminando $\rho$ y $\bm j$ de la expresión
    para $-\bm f$ usando la ley de Gauss y la ley de Ampère Maxwell y
    llevarla a la forma de ley de conservación empleando identidades
    del cálculo vectorial y las otras dos ecuaciones de Maxwell.
  \item El resultado es
    $$
    \bm g=\frac{1}{4\pi c}\bm E\times\bm B=\frac{1}{c^2}\bm S
    $$
    y
    $$
    T_{ij}=\frac{1}{4\pi}\left(E_iE_j+B_iB_j-\frac{1}{2}(E^2+B^2)\delta_{ij}\right).
    $$
  \end{itemize}

\question\label{19} Considera una carga $\pm Q$ depositada sobre las
  placas de un capacitor formado por dos discos metálicos de radio $a$
  separados una distancia $d\ll a$. Calcula la fuerza entre los
  discos.

  Pista:
  \begin{itemize}
  \item El resultado {\em no} es simplemente $\bm E Q$ con $\bm E$ el
    campo en el capacitor. Esto se debe a que el campo es discontinuo
    en la superficie donde se acumula la carga.
  \item Calcula la fuerza integrando el tensor de esfuerzos en una
    superficie que rodee la cara interna de uno de los discos, i.e.,
    un cilindro con una tapa en el espacio entre las placas y la otra
    tapa en el interior de la placa metálica.
  \item El resultado es consistente con un estado de {\em tensión} (lo
    opuesto a una presión) a lo largo de las líneas de campo
    $(1/8\pi)E^2$.
  \end{itemize}


\question\label{20} Considera una bobina en forma de cilindro de longitud $\ell$
  y radio $a\ll\ell$ en el que se enrollan uniformemente $N\gg1$
  vueltas de alambre que lleva una corriente $I$. Considera un pequeño
  pedazo de bobina contenida en un pequeño cilindro gaussiano de
  sección transversal {$\mathcal S$}
  con eje normal a la pared de la bobina, con una tapa en su interior
  y otra en el exterior. Demuestra que la fuerza sobre dicho pedazo de
  bobina es $P\mathcal S$ donde $P=(1/8\pi)B^2$ juega el papel de una
  presión, como si las líneas de campo estuvieran en un estado de
  compresión en su dirección normal.

  Nota:
  \begin{itemize}
  \item Si la corriente oscila 60 veces por segundo, esta presión
    sobre las paredes de la bobina oscila 120 veces por segundo. Esta
    presión variable es el origen del zumbido que se escucha en la
    cercanía de transformadores de potencia en la red eléctrica.
  \end{itemize}

\question \label{24} Considera una onda plana monocromática
  linealmente polarizada viajando en la dirección $z$ con amplitud
  $E_0$ la cual es totalmente absorbida por una placa fija de cierto material
  que ocupa la región $z>0$. Calcula
  \begin{parts}
  \part la potencia transmitida a la placa por unidad de área y
  \part la presión de radiación ejercida sobre la placa,
    promediadas sobre un ciclo de oscilación del campo.
  \part ¿Cómo cambiarían los resultados previos si la placa fuera un
    reflector perfecto en lugar de un absorbedor perfecto?
\end{parts}

  Pistas:
  \begin{itemize}
  \item Calcula el vector de Poynting y el tensor de esfuerzos.
  \item Promedialos en el tiempo.
  \item Si usas la notación compleja hay que tener cuidado, pues si
    $a(t)=\text{Re}\, a_0 e^{-i\omega t}$
    y $b(t)=\text{Re}\, b_0 e^{-i\omega t}$, entonces $a(t)b(t)\ne
    \text{Re}\, a_0 b_0 e^{-2i\omega t}$. Sin embargo, $\langle a(t)
    b(t)\rangle = \frac{1}{2}\text{Re}\, a_0 b_0^*=\frac{1}{2}\text{Re}\, a_0^* b_0$.

  \end{itemize}



\end{questions}
\end{document}
