\documentclass{exam}
\usepackage{espanhol}
\usepackage{bm}
\usepackage{amsmath}
\usepackage{comment}
%\usepackage{pstricks}
\decimalpoint
\pointname{\ puntos}
\cfoot[]{\thepage}
\begin{document}
\begin{center}
  \bf\large Tarea 6\\
  Propedéutico Electrodinámica\\
  2020-04-26\\[20pt]
\end{center}
\hbox to \textwidth{Nombre: \enspace\hrulefill}
Entregar el viernes 2020-05-01.

\begin{questions}
  \question Considera un fragmento de un circuito consistente en un
  conductor en forma de cilindro sólido
  con longitud $L$ y sección transversal $A$. Demuestra que en el
  mismo circula una corriente $I=V/R$ donde $V$ es el voltaje a través
  de los extremos del alambre, $R=L/A\sigma$ la
  resistencia del alambre y $\sigma$ la conductividad.

  Pistas:
  \begin{itemize}
  \item En una tarea previa vimos que en un conductor la densidad de
    corriente es $\bm j=\sigma\bm E$, y con un modelo simple obtuvimos
    $\sigma=ne^2\tau/m$, con $n$ el número de electrones por unidad de
    volumen, $e$ la carga del electrón, $\tau$ el tiempo que el
    electrón se mueve antes de chocar y $m$ la masa electrónica.
  \item La corriente $I$ se puede escribir en términos de $\bm j$ y la
    sección transversal $A$.
  \item El voltaje $V=\phi_a-\phi_b$ es la caida de potencial $\phi$
    entre los extremos $a$ y $b$ del alambre.
  \item El voltaje se puede escribir en términos del campo eléctrico
    $\bm E$ en el interior del alambre.
  \end{itemize}
  Notas:
  \begin{itemize}
  \item El campo eléctrico dentro de un conductor es cero en el caso
    estático.
  \item Argumenta por qué puede ser distinto de cero en el caso no
    estático.
  \item Argumenta por qué en este problema podemos considerar al campo
    eléctrico como uniforme en el interior del cilindro.
  \item La expresión $V=RI$ se conoce como la {\em Ley de Ohm}.
  \item Para geometrías más complicadas se cumple la ley de Ohm, pues
    la densidad de corriente es lineal en el campo y el voltaje
    también, y $R\propto 1/\sigma$, pero la dependencia en los parámetros
    geométricos puede ser más compleja.
  \item Se conoce a $\rho=1/\sigma$ como la resistividad. (No confunda
    estas $\rho$ y $\sigma$ con la densidad de carga volumetrica y
    superficial, aunque se suelan denotar con las mismas letras).
  \item Nota que a un material con una conductividad relativamente alta
    corresponde  una resistencia relativamente baja.
  \item Un resistor suele consistir de un matrial de baja
    conductividad conectada a través de terminales metálicas de alta
    conductividad.
  \item Para efectos prácticos suele aproximarse a
    las terminales metálicas, buenos conductores, por elementos con
    resistencia nula, equipotenciales. Las caidas de potencial se dan,
    sobre todo, a través de los materiales de baja conductividad.
  \end{itemize}

  \question Considera un sistema formado por un cilindro metálico de
  radio $a$ en contacto con un cilindro coaxial
  hueco de radio interior $a$ y radio exterior $b$ hecho de un
  material con conductividad $\sigma$ en contacto con
  otro cilindro coaxial metálico de radio interior $b$, todos de
  altura $L\gg b$. Calcula la resistencia $R$ entre el conductor
  interior y el exterior.

  Pistas:
  \begin{itemize}
  \item Argumenta por qué podemos tomar a los dos cilindros metálicos
    como equipotenciales, con potenciales $\phi(a)=\phi_a$ y
    $\phi(b)=\phi_b$.
  \item Argumenta por qué el potencial $\phi(\bm r)$ puede escribirse
    como $\phi(r)$, i.e., sin depender de la coordenada axial ni la
    coordenada angular.
  \item Demuestra que $\nabla^2\phi=0$ en $a<r<b$.
  \item Obtén el potencial $\phi(r)$ a partir de su ecuación
    diferencial y sus condiciones de contorno.
  \item Obtén el campo eléctrico $\bm E$.
  \item Obtén la densidad de corriente $\bm j$.
  \item Obtén la corriente total $I$.
  \item Obtén la resistencia.
  \end{itemize}

  \question  Considera un circuito en el que varios conductores
  confluyen sobre cierto nodo. Llama $I_n$ a la corriente que cada
  conductor lleva {\em hacia} dicho nodo. Demuestra la Ley de Kirchoff
  de los nodos, i.e., en el caso estacionario
  $$
  \sum_n I_n=0.
  $$
  Pistas:
  \begin{itemize}
  \item Encierra al nodo en el interior de una superficie cerrada
    $\mathcal S$.
  \item Argumenta por qué, en el caso estacionario,
    $$
    \int_{\mathcal S} d\bm a\cdot\bm j(\bm r)=0.
    $$
  \item Argumenta cómo se puede reescribir la integral anterior como
    una suma de integrales sobre secciones transversales de cada
    alambre que incide en dicho nodo.
  \item Relaciona cada una de estas integrales con la corriente $I_n$
    correspondiente.
  \end{itemize}
  Notas:
  \begin{itemize}
  \item Una corriente negativa {\em hacia} un nodo es una corriente
    positiva {\em desde} dicho nodo.
  \end{itemize}

  \question Considera un circuito y considera un bucle cerrado con $N$
  elementos $n=1\ldots N$ que van del nodos $n-1$ al nodo $n$,
  identificando al nodo $N$ con el nodo $0$. Denota con $V_n$ la caida
  de potencial a través del nodo $n$ yendo del nodo $n-1$ al $n$. Demuestra la ley de Kirchoff de
  los bucles
  $$
  \sum_n V_n=0
  $$
  Pistas:
  \begin{itemize}
  \item Identifica $V_n=\phi_{n-1}-\phi_{n}=\int_{\mathcal C_n}d\bm
    l\cdot\bm E(\bm r)$, donde $\mathcal C_n$ es una trayectoria que
    va del nodo $n-1$ al nodo $n$.
  \item Identifica $\sum_n V_n=\oint_{\mathcal C}d\bm l\cdot\bm E$ con
    una integral cerrada sobre una
    trayectoria cerrada $\mathcal C=\mathcal C_1$ seguida de $\mathcal C_2$  seguida de
    $\mathcal C_3\ldots\mathcal C_N$.
  \item El resultado se sigue de que el campo eléctrico es conservativo.
  \end{itemize}
  Notas:
  \begin{itemize}
  \item Una caida de potencial negativa es una subida de potencial.
  \item Las dos leyes de Kirchoff pueden emplearse para plantear
    sistemas lineales de ecuaciones que describen las corrientes y caidas
    de potencial en circuitos arbitrarios formados por fuentes de
    voltaje (subidas de potencial por cantidades dadas) y resistencias
    (caidas de potenciales de acuerdo a las leyes de Ohm).
  \item También se pueden usar para plantear ecuaciones diferenciales
    que cumplen los voltajes y corrientes en circuitos con
    capacitancias e inductancias.
  \end{itemize}

  \question Considera un circuito que contiene $N$
  resistencias $R_n$, $n=1\ldots N$ conectadas en serie entre los
  nodos $a$ y $b$, es decir, una detrás de
  otra, con $R_n$ entre los nodos $n-1$ y $n$, identificando el nodo
  $0$ con el nodo $a$ y el nodo $N$ con el nodo $b$ y sin nada
  más conectado a los nodos intermedios $n=1\ldots N-1$. Demuestra que
  este circuito es equivalente a otro circuito en el que las resistencias $R_n$
  son reemplazadas por una sola resistencia efectiva $R_{\mathrm{ef}}$
  conectada entre $a$ y $b$ y dada por
  $$
  R_{\mathrm{ef}}=\sum_n R_n.
  $$
  \question Considera un circuito que contiene $N$
  resistencias $R_n$, $n=1\ldots N$ conectadas en paralelo entre los
  nodos $a$ y $b$, es decir, todas con una terminal conectada al nodo
  $a$ con y la otra terminal conectada al nodo $b$. Demuestra
  que es equivalente a otro circuito en el que las resistencias $R_n$
  son reemplazadas por una sola resistencia efectiva $R_{\mathrm{ef}}$
  dada por
  $$
  \frac{1}{R_{\mathrm{ef}}}=\sum_n\frac{1}{R_n}.
  $$

  \question Considera un capacitor con capacitancia $C$ cuyas
  terminales se conectan a las terminales de un resistor con resistencia $R$. Al
  tiempo $t=0$ una placa (digamos $a$) del capacitor tiene carga $Q_0$ (y la
  otra, digamos $b$, tiene carga $-Q_0$).
  \begin{parts}
    \part Calcula la carga $Q(t)$ en la placa $a$ del capacitor para tiempos
    subsecuentes $t>0$.
    \part Calcula la caida de potencial $V(t)$ a través del capacitor.
    \part Calcula la corriente $I(t)$ a través de la resistencia.
  \end{parts}

  Pistas:
  \begin{itemize}
  \item Argumenta por qué la caida de potencial $V=Q/C$ a través desde
    la terminal $a$ a la terminal $b$ del capacitor es la misma que la
    caida de potencial $V=RI$ a través de las terminales
    correspondientes del resistor, donde convenimos que $I$ es la
    corriente de $a$ a $b$ a través del resistor.
  \item Argumenta por qué con las definiciones arriba, $I=-dQ/dt$.
  \item Convierte la ley de Kirchoff en una ecuación diferencial.
  \item Resuélvela.
  \item Aplica condiciones iniciales.
  \end{itemize}
  Notas:
  \begin{itemize}
  \item La carga y corriente decaen exponencialmente con un tiempo
    característico $\tau=RC$, más largo mientras más capacitancia (más
    carga almacenada para un voltaje dado) y más resistencia (menos
    corriente para el mismo voltaje).
  \end{itemize}

  \question \label{l:a}Considera un campo magnético $\bm B(\bm r)=\alpha
  x\hat{\bm z}$ que depende de $x$ y apunta en dirección $z$, con
  $\alpha$ constante. Considera un circuito $\mathcal C$ formado
  por un alambre en forma de espira cuadrada de lado $L$ con lados
  paralelos a los ejes $x$ y $y$
  y que se desplaza con velocidad $\bm v=v\hat{\bm x}$ constante en la
  dirección $x$. Calcula la {\em fuerza electromotriz} $\mathcal E$
  alrededor de este circuito.

  Pistas:
  \begin{itemize}
  \item Se define la fuerza electromotriz como $\oint_{\mathcal C}d\bm
    l\cdot \bm F/q$ donde $\bm  F$ es la fuerza que sentiría una carga $q$
    en el circuito, i.e., es el trabajo virtual que el campo haría
    sobre una carga al dar una vuelta al circuito, por unidad de carga.
  \item En este caso la fuerza sería la fuerza de Lorentz producida
    por el campo magnético $\bm F=q\bm v\times \bm B/c$.
  \end{itemize}
  Notas:
  \begin{itemize}
  \item Si el circuito tuviera una resistencia $R$, la fuerza
    electromotriz establecería una corriente $I=\mathcal E/R$, como si
    el circuito se cerrara a través de una fuenta de voltaje
    $V=\mathcal E$.
  \end{itemize}

  \question Considera un campo magnético $\bm B(t)=\eta t\hat{\bm
    z}$ que apunta en dirección $z$ y que depende del
  tiempo, donde $\eta$ es constante. Considera un circuito $\mathcal C$ formado
  por un alambre en forma de espira cuadrada de lado $L$ que descanza
  sobre el plano $xy$ con lados paralelos a los ejes $x$ y
  $y$. Calcula la fuerza electromotriz.

  Pistas:
  \begin{itemize}
  \item De acuerdo a la {\em Ley de Inducción} de Faraday, alrededor
    de un circuito que no se mueve
    $$\mathcal E=-\frac{1}{c}\frac{d}{dt}\Phi_B,$$
    donde
    $$\mathcal E=\int_{\partial \mathcal A} d\bm l\cdot\bm E$$
    es la fuerza electromotriz a lo largo de la frontera
    $\partial\mathcal A$ de una superficie orientable $\mathcal A$, y
    $$\Phi_B=\int_{\mathcal A}d\bm a\cdot\bm B$$
    es el {\em flujo magnético} a través de dicha superficie.
  \end{itemize}

  Notas:
  \begin{itemize}
  \item Compara la solución de los dos problemas anteriores en el caso
    $\eta=v\alpha$.
  \item ¿Cómo se vería el problema \ref{l:a} en un sistema de
    referencia en que el circuito estuviera fijo? (suponiendo
    velocidades $v\ll c$ bajas).
  \item Einstein notó que la respuesta a la nota anterior es general. La fuerza
    electromotriz debida al movimiento de un circuito con fuentes
    fijas es igual a la fuerza electromotriz sobre un circuito fijo
    debida a una fuente en movimiento si la velocidad relativa es la
    misma en ambos casos. Einstein observó que esto no
    podía ser una coincidencia y por ello postuló el {\em Principio de
      Relatividad}.
  \item Es común resolver problemas que involucran a la fuerza de
    Lorentz como si fueran problemas que involucran inducción. Aunque
    no es del todo correcto, gracias al principio de
    relatividad lleva a la solución correcta.
  \item Un ejemplo puede ser el siguiente problema.
  \end{itemize}

  \question Considera dos barras $b_1$ y $b_2$ conductoras largas paralelas
  unidas entre sí por una barra $b_3$ conductora corta de longitud $L$
  formando una figura como la letra $U$. Se coloca $b_3$
  horizontalmente y se inclinan las barras largas hasta que forman un
  ángulo $\theta$ con la vertical. Se cierra el circuito
  con una barra $b_4$ parcialmente conductora de longitud $L$, masa $M$ y
  resistencia $R$ colocada horizontalmente sobre las barras $b_1$ y
  $b_2$ con las que hace contacto eléctrico y sobre las que puede deslizar
  libremente. Calcula la velocidad a la que se desliza la barra $b_4$
  bajo la acción de la gravedad $g$ en presencia de un campo magnético
  vertical $B$.

  Pistas:
  \begin{itemize}
  \item Cuando la barra $b_4$ se mueve, disminuye su energía potencial
    gravitacional y adquiere energía cinética.
  \item Cuando la barra se desliza con una velocidad $v$ aparece una fuerza electromotriz
    $\mathcal E$ que produce una corriente $I=\mathcal E/R$ en el
    circuito $b_1$-$b_3$-$b_2$-$b_4$ proporcional a la $v$.
  \item  Esta corriente disipa energía al atravesar la resistencia. La
    potencia disipada es $\mathcal P=RI$ y es mayor mientras más
    rápido baja la barra.
  \item Cuando la energía disipada por unidad de tiempo iguala a la
    energía potencial perdida por unidad de tiempo, la barra ya no
    puede incrementar su energía cinética y llega a una velocidad
    terminal.
  \end{itemize}

  \question \label{l:b} Considera una bobina formada por un alambre enrollado
  uniformemente alrededor de un cilindro de radio $a$ y longitud $\ell\gg
  a$ formado de un material con permeabilidad $\mu$ dando $N\gg 1$
  vueltas. Se hace circular por el alambre una corriente
  $I(t)$. Calcula la caida de potencial $V$ a través de las terminales
  de la bobina.

  Pistas:
  \begin{itemize}
  \item Utiliza la ley de Ampère para obtener el campo magnético $\bm H(t)$ en el
    interior del cilindro ignorando efectos de borde.
  \item Calcula la densidad de flujo magnético $\bm B$ en el interior
    del cilindro.
  \item Considera un circuito que va de un punto $a$ en una terminal
    de la bobina a través del alambre y siguiendo a la corriente hasta
    un punto $b$ en la terminal opuesta y regresa al punto $a$
    cerrando el circuito a través del espacio vacío por fuera de la
    bobina.
  \item Calcula el flujo magnético $\Phi_B$ a través de este
    circuito.
  \item Calcula la fuerza electromotriz $\mathcal E$ a lo largo de este circuito.
  \item Argumenta que el campo eléctrico en el interior del alambre es
    nulo, por ser buen conductor.
  \item Por tanto podemos identificar $\mathcal E=\int_b^a d\bm
    l\cdot\bm E$ con la contribución a la integral cerrada del tramo
    entre terminales, afuera de la bobina y afuera del
    alambre.
  \item Afuera de la bobina el campo magnético es (prácticamente)
    nulo.
  \item Por tanto, afuera de la bobina el campo eléctrico es
    conservativo y
  \item se puede definir un potencial.
  \item Compara la integral anterior con la caida de potencial $V$.
  \item Obtén $V$.
  \end{itemize}
  Notas:
  \begin{itemize}
  \item Podemos escribir $V=L d I/dt$ con
    $$
    L=\frac{4\pi^2}{c^2} \mu a^2\frac{N^2}{L}
    $$
    la {\em autoinductancia} de la bobina.
  \item Para otras geometrías también tendríamos $V=L d I/dt$, pero
    con otras expresiones para la inductancia.
  \item Las unidades de la inductancia son aceleración inversa.
  \end{itemize}


  \question Considera un inductor con autoinductancia $L$ en el que
  circula una corriente $I$. Demuestra que la energía que almacena es
  $$
  U=\frac{1}{2}LI^2.
  $$

  Pistas:
  \begin{itemize}
  \item Cuando una carga $\delta q$ pasa de una terminal $a$ a la otra
    terminal $b$ de un inductor a través del cual hay una caida de
    potencial $V$, pierde una energía $\delta U=V\delta q$. Esta
    energía se almacena en la bobina como energía magnética.
  \item La potencia transferida a la bobina es entonces $\mathcal
    P=VI$ donde $I=\delta q/\delta t$ es la carga que circula por
    unidad de tiempo.
  \item Como $V=L dI/dt$, $\mathcal P=LIdI/dt$.
  \item La energía almacenada es $U(t)=\int^t_{t_0} dt'\,\mathcal
    P(t')$ inciando la integral cuando la
    corriente es nula.
  \end{itemize}
  Notas:
  \begin{itemize}
  \item Para un sistema con muchos circuitos podemos generalizar los
    resultados anteriores y escribir
    $V_n=\sum_m L_{nm}dI_m/dt$ donde $L_{nm}$ es la autoinductancia
    del circuito $n$ si $n=m$ y la inductancia mutua entre los
    circuitos $n$ y $m$ si $n\ne m$.
  \item La inductancia mutua se debe a que una corriente $I_m$ en el
    circuito $m$ produce en general un campo $B_m$ que puede atravesar
    al circuito $n$.
  \item La energía total de un sistema de inductores es
    $$U=\frac{1}{2}L_{nm}I_nI_m.$$
  \item Podemos invertir nuestro razonamiento y emplear esta ecuación
    para definir la autoinductancia y la inductancia mutua y de aquí
    concluir que $V_n=\sum_m L_{nm}dI_m/dt$.
  \end{itemize}

  \question Muestra que la energía en el inductor del problema
  \ref{l:b} puede escribirse como
  $$
  U=\int_V d^3r\, u(\bm r),
  $$
  donde $u(\bm r)$ es la densidad de energía magnética
  $$
  u=\frac{\bm B\cdot\bm H}{8\pi}.
  $$

  Pistas:
  \begin{itemize}
  \item Obtén $U$ en términos de la corriente $I$ empleando la
    autoinductancia.
  \item Escribe la corriente en términos de $B$.
  \item Escribe la energía total en términos de $B$.
  \item Divide entre el volumen de la bobina $\Omega=\ell\pi a^2$.
  \end{itemize}
  Notas:
  \begin{itemize}
  \item Esta demostración es muy limitada, pero el resultado es
    totalmente general.
  \item Una forma de calcular la inductancia de un circuito es primero
    calcular la energía a partir de su densidad y después escribirla
    en términos de la corriente.
  \end{itemize}

  \question Considera un cable coaxial de longitud $\ell$ formado por un cilindro hueco
  de paredes delgadas y de radio $a\ll \ell$ en el que va una corriente
  $I$ rodeado por otro cilindro delgado coaxial de radio $b>a$, $b\ll \ell$ por el
  que va la corriente de retorno $-I$. Calcula la autoinductancia $L$
  del sistema.

  Pistas:
  \begin{itemize}
  \item Calcula el campo $\bm B(\bm r)$ entre ambos cilindros
    ignorando efectos de borde.
  \item Argumenta por qué se puede ignorar el campo en el resto del espacio.
  \item Calcula la densidad de energía $u(\bm r)=u(r)$.
  \item Integrala sobre el volumen entre los cilindros para obtener la
    energía total $U$.
  \item Identifica la autoinductancia $L$.
  \end{itemize}

  \question Considera un circuito formado por un capacitor con
  capacitancia $C$ cuyas terminales $a$ y $b$ están conectadas a las
  terminales de un inductor con autoinductancia $L$. Al tiempo $t=0$
  la carga en el electrodo $a$ del capacitor es $Q(0)=Q_0$ (y en el
  electrodo $b$ es $-Q_0$) y la corriente a través del inductor de $a$
  hacia $b$ es $I(0)=I_0=0$.
  \begin{parts}
    \part Calcula la carga $Q(t)$ en el electrodo $a$.
    \part Calcula la corriente $I(t)$ de $a$ a $b$ a través del
    inductor.
    \part Calcula la caida de potencial $V(t)$ de $a$ a $b$.
  \end{parts}

  Pistas:
  \begin{itemize}
  \item Argumenta mediante la ley de Kirchoff que la caida de potencial $V(t)$ de
    $a$ a $b$ a través del capacitor es la misma que a
    través del inductor.
  \item Relaciona dicha caida con la carga $Q(t)$ e $I(t)$.
  \item Escribe $I(t)$ en términos de $Q(t)$ (ojo con los signos y la
    dirección de la corriente).
  \item Escribe una ecuación diferencial para $Q$.
  \item Resuelvela imponiendo condiciones iniciales.
  \end{itemize}
  Notas:
  \begin{itemize}
  \item Este circuito es un oscilador. La energía electrica
    almacenada en el capacitor cargado se convierte en energía
    magnética almacenada en el inductor conforme la carga disminuye y
    la corriente aumenta, y luego regresa al capacitor cuando se
    vuelve a cargar con signo opuesto y disminuye la corriente.
  \item La frecuencia de oscilación de la carga, corriente y voltaje
    es $\omega=1/\sqrt{LC}$.
  \item La frecuencia con que se intercambia la energía es el doble.
  \end{itemize}

  \question Demuestra que en un circuito en que todos los voltajes,
  cargas y corrientes oscilan con una misma frecuencia $\omega$ podemos
  reemplazar las resistencias $R$,  capacitancias $C$ e inductancias
  $L$ por {\em impedancias} $Z$ que cumplen una ley de Ohm
  generalizada $V=ZI$, donde
  \begin{parts}
    \part $Z_R=R$,
    \part $Z_C=i/\omega C$,
    \part $Z_L=-i\omega L$.
  \end{parts}

  Pistas:
  \begin{itemize}
  \item Escribe el voltaje, carga y corriente como
    $V(t)=\mathrm{Re}\,\tilde V(t)=\mathrm{Re}\,
    V_0 \exp(-i\omega t)$, $Q(t)=\mathrm{Re}\,\tilde Q(t)=\mathrm{Re}\,
    Q_0 \exp(-i\omega t)$, $I(t)=\mathrm{Re}\,\tilde I(t)=\mathrm{Re}\,I_0 \exp(-i\omega t)$.
  \item La corriente $I(t)$ que {\em entra} a un capacitor se
    relaciona con
    la carga $Q$ almacenada en el electrodo correspondiente
    $I=dQ/dt$.
  \item Luego, $\tilde Q=i\tilde I/\omega=C \tilde V\Rightarrow \tilde
    V=Z_C\tilde I$.
  \item Similarmente, $V=Ld I/d t$ se puede escribir como $\tilde
    V=Z_L\tilde I$.
  \end{itemize}
  Notas:
  \begin{itemize}
  \item La solución estacionaria a las ecuaciones correspondientes a
    un circuito con fuentes de poder oscilatorias con frecuencia
    $\omega$, resistencias, capcitancias e inductancias se
    puede obtener de un circuito con fuentes e impedancias.
  \item Cada impedancia se puede tratar como si fuese una simple
    resistencia, pero con valores complejos.
  \item En particular, se pueden sumar impedancias en serie y en
    paralelo.
  \item Como resultado, se pueden obtener impedancias complejas.
  \item La parte real de una impedancia es la parte disipativa y la
    parte imaginaria es la parte reactiva; una produce disipación de
    energía y la otra un defasamiento entre voltaje y corriente.
  \end{itemize}

  \question Demuestra que un conjunto de capacitores con impedancias
  $C_n$, $n=1\ldots N$ se pueden cambiar,
  \begin{parts}
    \part si están conectadas en paralelo, por una capacitancia efectiva
    $$C_{\mathrm{ef}} =\sum_n C_n;$$
    \part si están conectadas en serie, por una capacitancia efectiva
    $C_{\mathrm{ef}}$ donde
    $$
    \frac{1}{C_{\mathrm {ef}}}=\sum_n\frac{1}{C_n}.
    $$
  \end{parts}
  Pistas:
    \begin{itemize}
    \item Suma las impedancias en serie y en paralelo como si fueran
      resistencias.
    \item Escribelas en términos de la capacitancia.
    \end{itemize}

    \question ¿Cómo se suman inductancias en serie y en paralelo?

    \question Considera una fuente de voltaje $V_e(t)=V_0\cos\omega t$
    conectada a través de una resistencia $R$ a un capacitor con
    capacitancia $C$.
    Encuentra el voltaje de salida $V_s(t)$ a través del capacitor a tiempos largos.

    Pistas:
    \begin{itemize}
    \item Argumenta por qué a tiempo largos $t>RC$ la solución es una
      simple oscilación con frecuencia $\omega$.
    \item Reemplaza el circuito por un divisor de voltaje formado por
      una fuente y dos impedancias $Z_R$ y $Z_C$.
    \item Por tanto el voltaje {\em complejo} de salida es
      $$\tilde V_s=\frac{Z_C}{Z_C+Z_R}\tilde V_e.$$
    \end{itemize}

    Notas:
    \begin{itemize}
    \item Nota que a frecuencias bajas $\omega\ll 1/RC$ el voltaje de
      salida es igual al de entrada $V_s=V_e$.
    \item Nota que a frecuencias altas el voltaje de salida es muy
      pequeño y decae en proporción inversa a la frecuencia
      $$\tilde V_s=\frac{i}{\omega RC} \tilde V_e.$$
    \item Este circuito se conoce como {\em filtro pasabajos}.
    \item Análogamente se pueden diseñar filtros pasabajos, pasaaltos
      y pasabandas con redes de resistencias, capacitores  y/o inductores.
    \end{itemize}

    \question Considera un dipolo magnético $\bm m(t)=m(\cos\omega t,
    \sin\omega t,0)$ que gira alrededor del eje $z$. Considera una
    espira circular de radio $a$ colocada paralela al plano $yz$ y
    centrada alrededor del eje $x$ a una distancia $x=d\gg a$ del origen.
    \begin{parts}
      \part Calcula la fuerza electromotriz $\mathcal E(t)$ alrededor
      de la espira.
      \part Supón que se abre la espira y se conectan sus extremos a través
      de una resistencia $R$. Calcula, la corriente eléctrica a través
      de la resistencia.
      \part Calcula la potencia disipada en la resistencia.
      \part Calcula la torca sobre el dipolo rotante.
      \part Calcula la potencia que se le debe proporcionar para
      mantener su rotación.
    \end{parts}

    Pistas:
    \begin{itemize}
    \item Usa la condición $a\ll d$ para simplificar el cálculo.
    \item La corriente en la espira produce un dipolo inducido orientado en $x$
      que varía en el tiempo.
    \item Este dipolo genera un campo magnético que interacciona con
      el dipolo rotatorio en el centro.
    \end{itemize}
    Notas:
    \begin{itemize}
    \item Este es una versión simplificada de un generador de voltaje
      de {\em corriente alterna}.
    \item Una variante es que sea la espira la que gira alrededor del
      dipolo y que sus extremos se conmuten, para generar una
      contribución de corriente directa.
    \item La condición $a\ll d$ no es esencial, pero simplifica el
      cálculo.
    \item Se verifica la conservación de la energía, i.e., la energía
      disipada en la resistencia iguala a la energía propocionada al
      dipolo rotante mediante la torca que se le debe aplicar para
      evitar que se frene.
    \end{itemize}

    \question Considera tres pares de bobinas de Helmholtz idénticas centradas
    en el origen con ejes en el plano $xy$, formando ángulos
    de $\theta_1=0^\circ$, $\theta_2=120^\circ$ y $\theta_3=240^\circ$
    con respecto al eje $x$. Se excitan dichas bobinas con corrientes
    $I_1=A\cos(\omega t)$, $I_2=A\cos(\omega t-2\pi/3)$ e $I_3=A\cos(\omega t+2\pi/3)$.
    En el origen se coloca un pequeño imán con momento magnético $\bm
    m$ en el plano $xy$ y libre de girar alrededor del eje $z$ sujeto
    a un poco de fricción.
    \begin{parts}
      \part Muestra que cerca del origen el campo $\bm B(t)$ que producen las bobinas
      en el origen tiene magnitud constante y rota con velocidad
      angular constante $\omega$ alrededor del eje $z$.
      \part Muestra que el dipolo giraría a tiempos largos con la misma velocidad
      angular $\omega$ y alineado con el campo magnético.
      \part Muestra que si aplicamos una torca externa $\bm\tau$
      constante y no demasiado grande al dipolo para intentar frenar su rotación, este seguiría
      rotando con la misma velocidad angular pero su dirección
      formaría un ángulo de retraso con respecto a $\bm B$.
      \part Para este caso, calcula la energía que el campo magnético
      $\bm B(t)$ proporciona al dipolo para conservar su rotación.
      \part Calcula la fuerza electromotriz que el campo producido por
      el dipolo produce en las bobinas.
      \part Calcula la potencia que es necesario proporcionar a las
      bobinas para conservar la corriente eléctrica en ellas.
    \end{parts}

    Pistas:
    \begin{itemize}
    \item La torca sobre un dipolo fue calculada en una tarea previa.
    \end{itemize}

    Notas:
    \begin{itemize}
    \item Esta es una versión simplificada de un motor eléctrico
      trifásico de velocidad constante.
    \item La energía se conserva. La potencia eléctrica que se
      transmite a las bobinas es igual a la potencia que se transmite
      al dipolo rotatorio y es igual al trabajo mecánico que dicho
      dipolo ejerce sobre los dispositivos sujetos al eje de giro.
    \item La aproximación dipolar no es esencial, pero simplifica los
      cálculos.
    \item El generador (problema anterior) y el motor eléctrico (este problema)
      forman la base de la enorme industria electromecánica en que
      se ha sustentado el desarrollo industrial del último siglo.
    \end{itemize}





\end{questions}
\end{document}
