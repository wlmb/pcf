\documentclass{exam}
\usepackage{espanhol}
\usepackage{bm}
\usepackage{amsmath}
\usepackage{comment}
%\usepackage{pstricks}
\decimalpoint
\pointname{\ puntos}
\begin{document}
\begin{center}
\bf\large Tarea 3\\
Propedéutico Electrodinámica\\
\date{2020-03-10}
\today\\[20pt]
\end{center}
\hbox to \textwidth{Nombre: \enspace\hrulefill}
Entregar el viernes 2020-04-10.

\begin{questions}
  \question Demuestra que para una función escalar $f(\bm r)$ de la posición
  $\bm r=(x,y,z)$ suficientemente suave, diferenciable, etc., y un
  vector $\bm h=(h_x, h_y, h_z)$ suficientemente pequeño,
  podemos hacer la expansión de {\em Taylor}
  \begin{align*}
    f(\bm + \bm h)=&f(\bm r)+\bm h\cdot\nabla f(\bm r)+\frac{1}{2}(\bm
                     h\cdot\nabla)^2 f(\bm r)+\ldots\\
    =&\sum_n \frac{1}{n!}(\bm h\cdot\nabla)^n  f(\bm r)\\
    =& \exp(\bm h\cdot\nabla)f(\bm r).
  \end{align*}

  Pista:
  \begin{itemize}
  \item Considera la función $g(t)\equiv f(\bm r+t\bm h)$ escalar de
    una variable real $t$ y haz una expansión ordinaria de Taylor
    respecto a $t$, usando la regla de la cadena.
  \end{itemize}
  Nota:
  \begin{itemize}
  \item Identifica en la última expresión $\exp(i\bm
    h\cdot\hat{\bm p}/\hbar)$, con $\hat{\bm p}$ el operador ímpetu de
    la mecánica cuántica, el {\em generador de translaciones}.
  \end{itemize}
  \question Considera una carga distribuida con una densidad $\rho(\bm
  r)$ en una región de tamaño $D$ alrededor del origen. Demuestra que
  el potencial $\phi(\bm r)$ lejos de dicha región $r\gg D$ se puede
  escribir como
  $$\phi(\bm r)=\frac{Q}{r} + \frac{\bm p\cdot\bm r}{r^3} +\ldots
  $$
  donde
  $Q=\int d^3r\,\rho(\bm r)$ es la carga total de la distribución,
  $\bm p=\int d^3r \rho(\bm r)\bm r$ es el momento dipolar, dado por
  el primer {\em momento} de la distribución.

  Pistas:
  \begin{itemize}
  \item Escribe el potencial Coulombiano y haz una expansión de Taylor
    para el kernel $1/|\bm r-\bm r'|$ alrededor de $\bm r'=\bm 0$.
  \end{itemize}

  Notas:
  \begin{itemize}
  \item Generalmente, cada término es de orden $D/r$ comparado con el
    anterior.
  \item Si $Q\ne 0$, el primer término domina a distancias grandes.
  \item Si $Q=0$ pero $\bm p\ne \bm 0$, entonces el segundo término
    domina a distancias grandes.
  \item En general, el primer término distinto de cero es el único
    relevante a distancias suficientemente lejanas. De lejos, toda
    distribución de carga es indistinguible de una carga puntual, a
    menos que la carga total sea cero, en cuyo caso, todo sistema es
    un dipolo, a menos que el momento dipolar sea cero, en cuyo caso,
    todo sistema es un cuadrupolo\ldots
  \item ¿Cómo escribirías la definición de $Q$ y de $\bm p$ para cargas
    discretas?
  \end{itemize}

  \question Aplica el resultado anterior al de dos cargas $q$ y $-q$
  separadas una distancia $a$.

  \question Demuestra que el campo eléctrico producido por un dipolo
  $\bm p$ colocado en el origen es
  $$
  \bm E(\bm r)=\frac{3\bm p.\bm r\bm r-\bm p r^2}{4^5}.
  $$
  Pista:
  \begin{itemize}
  \item Usa $\bm = -\nabla\phi$.
  \end{itemize}
  Nota:
  \begin{itemize}
  \item Si el dipolo estuviera en una posición $\bm r_0$ basta cambiar
    $\bm r$ por $\bm r-\bm r_0$ en las expresiones anteriores.
  \end{itemize}

  \question Considera un cubo de lado $L$ orientado con los planos
  cartesianos, formado por formado por $N$ átomos, cada uno sin carga
  ni momento dipolar incialmente. En cierto momento un electrón de
  cada átomo se desplaza una distancia $a$ en la dirección $z$.
  \begin{parts}
    \part Demuestra que el número de electrones que atraviesan la cara
    superior de abajo hacia arriba es $Q=Na/L=na/L^2$, donde $n=N/L^3$
    es la densidad de número de los átomos.
    \part Muestra que cada átomo adquiriría un momento dipolar $\bm
    p=-e\bm a=-ea\hat{\bm z}$.
    \part Muestra que el momento dipolar por unidad de volumen sería
    $\bm P=n\bm p=\-ne\bm a$.
    \part Muestra que la carga que atraviesa la superficie superior es
    $Q=\bm P\cdot\hat {\bm z} L^2$.
    \part Demuestra que la carga que atraviesa las superficies
    laterales es nula.
    \part Generaliza este resultado y muestra que para cualquier
    pequeña superficie con área $da$ y normal $\hat{\bm n}$ colocada
    en la posición $\bm r$ dentro
    en un medio no polarizado, la carga que atraviesa la superficie en
    la dirección de $\hat n$
    cuando el sistema se polariza por el movimiento de algunas o todas sus
    cargas es $dQ=\bm P(\bm r)\cdot d\bm a$, donde $d \bm a=\hat{\bm n}da$ y
    $\bm P(\bm r)$ es el momento dipolar eléctrico por unidad de
    volumen evaluado en la superficie.
    \part Considera un volumen $V$ cuya frontera sea la superficie
    orientable $\partial V$ dentro de un material neutro no
    polarizado. Demuestra que la carga que sale del volumen a través
    de su superficie cuando el sistema se polariza es
    $$
    Q_{\partial V}=\int_{\partial V} d\bm a\cdot \bm P.
    $$
    \part Argumenta mediante la ley de conservación de la carga y el
    teorema de Gauss que
    el interior del volumen queda cargado con una carga
    $$
    Q_V=-\int_{\partial V} d\bm a\cdot \bm P=-\int_V d^3r\,
    \nabla\cdot\bm P.
    $$
    \part Argumenta que, como el volumen anterior es arbitrario, hay
    una densidad de carga $\rho^i$ asociada a la polarización dada por
    $$
    \rho^i(\bm r)=-\nabla\cdot\bm P(\bm r).
    $$
    \part Argumenta por qué, si el material es finito, en su
    superficie aparecerá una carga $\sigma^i$ por unidad de superficie
    dada por
    $$
    \sigma^i(\bm r)=\bm P(\bm r)\cdot\hat{\bm n}(\bm r),
    $$
    donde $\bm r$ se halla en la frontera del material, $\bm P(\bm r)$
    es la polarización justo adentro del material y $\hat{\bm n}(\bm
    r)$ es una normal que apunta desde adentro hacia afuera del
    material.
    \part Argumenta por qué en una interface entre dos medios (1 y 2)
    polarizados, hay una densidad de carga superficial debida a la
    polarización $P(\bm r)$ dada opr
    $$
    \sigma(\bm r)=(\bm P(\bm r_1)-\bm P(\bm r_2))\cdot \hat{\bm n}(\bm r),
    $$
    donde $\bm r$ está en la superficie, $\bm r_1=\bm r-\delta
    \hat{\bm n}$ es un vector pegado a $\bm r$ pero dentro del medio
    1, $\bm r_2=\bm r+\delta \hat{\bm n}$ es un vector pegado a $\bm
    r$ pero dentro del medio 2, tomamos el límite $\delta\to 0^+$ y
    el vector normal a la superficie $\hat{\bm n}(\bm r)$ apunta
    desde el medio 1 hacia el medio 2.
  \end{parts}
  \question Considera un cilindro circular recto de radio $R$ y altura
  $L$ uniformemente polarizado con polarización $\bm P$ a lo largo del
  eje. Calcula el momento dipolar total del cilindro $\bm p$
  \begin{parts}
    \part integrando su polarización sobre el volumen,
    \part a partir de las cargas superficiales empleando la definición de momento dipolar.
  \end{parts}
  \question Utiliza la ley de Gauss para demostrar la {\em ley de
    Gauss macroscópica}
  $$
  \nabla\cdot \bm D(\bm r)=4\pi\rho^e,
  $$
  donde $\bm D=E+4\pi P$ es el {\em desplazamiento} y
  $\rho^e$ es la densidad de carga externa.

  Pista:
  \begin{itemize}
  \item Escribe la densidad de carga total $\rho=\rho^e+\rho^i$ y
    escribe la densidad de carga {\em interna} o de polarización
    $\rho^i$ en términos de $\bm P$.
  \end{itemize}

  Nota:
  \begin{itemize}
  \item A veces se usan los superíndices $t$, $e$, e $i$ para denotar
    cargas totales, externas e internas, pero muchas veces se olvidan
    dichas anotaciones, esperando que el lector adivine por el
    contexto a qué tipo de carga se refiere uno. Si en un texto leen
    $\nabla\cdot \bm D=r\pi\rho$ deben entender que en este caso $\rho$
    se refiere a las cargas internas, mentras que si leen $\nabla
    E=4\pi\rho$, entonces deberán entender que $\rho$ se refiere a las
    cargas totales.
  \end{itemize}

  \question Considera un volumen $V$ dentro de un material. Demuestra
  que la cantidad de carga externa dentro del volumen está dada por
  $$
  Q^e_V=\int d\bm a\cdot\bm D(\bm r).
  $$

  \question Demuestra que en una superficie cualquiera que separa dos
  regiones 1 y 2 se cumple la condición
  $\bm D(\bm r_2)-\bm D(\bm r_1)\cdot\hat{\bm n}(\bm
  r)=4\pi\sigma^e(\bm r)$, donde $\bm r$ es una posición en la superficie, $\bm
  r_2$ y $\bm r_1$ son posiciones pagadas a $\bm r$ del lado 2 y del
  lado 1 de la superficie respectivamente, y donde $\sigma^e$ es la
  densidad superficial de carga depositada justo en la superficie.

  Pista:
  \begin{itemize}
  \item Integra la ley de Gauss en un pequeño cilindro con tapas
    paralelas a la superficie, una de cada lado, y toma el límite en
    que las paredes del cilindro se colapsan.
  \end{itemize}

  Nota:
  \begin{itemize}
  \item En ausencia de una carga externa superficial podemos concluir
    que la componente normal del desplazamiento es contínua $\Delta
    D_\perp(\bm r)=E_\perp(\bm r_2)-E_\perp(\bm 1)=(\bm E(\bm r_2)-\bm
    E(\bm 1))\cdot\hat{\bm n}(\bm r)$.
  \item La otra condición de contorno, la continuidad de $\bm E_\|$
    sigue sin cambio, así como las ecuaciones $\nabla\times\bm E=0$,
    $\int_{\partial A}d \bm l\cdot\bm E=0$, $\bm E=-\nabla\phi$ y la
    condición de continuidad de $\phi$.
  \end{itemize}

  \question Considera una placa en forma de prisma cuadrado de lado
  $L$ y altura $h\ll L$ que descansa sobre su base en el plano $xy$ y
  hecha de un material ferroeléctrico, i.e., que
  tiene una transición de fase que lo hace pasar espontáneamente de un
  estado con $\bm P=0$ a un estado con $\bm P\ne 0$. Calcula el campo
  eléctrico $\bm E$ en todo el espacio cuando el material se polariza,
  suponiendo que la polarización apunta en la dirección $z$ e
  ignorando efectos de borde.

  Pistas:
  \begin{itemize}
  \item Calcule las cargas de polarización y de ahí obtenga el campo
    eléctrico que producen.
  \end{itemize}

  \question Considera un modelo atómico clásico (malísimo, pero no importa) en
  que un electrón con carga $-e$ está ligado a al núcleo por un
  resorte con constante $\kappa$. La posición de equilibrio del electrón
  es {\em encima} del núcleo. Los demás electrones, por lo pronto,
  supongamos que son inertes, que no se mueven.
  \begin{parts}
    \part Demuestra que al aplicar un campo $\bm E$ estático el átomo adquiere un
    momento dipolar $\bm p=\alpha \bm E$, donde la polarizabilidad
    $\alpha$ es
    $$
    \alpha=\frac{e^2}{m\omega_0^2},
    $$
    $\omega_0^2=\kappa/m$ y $m$ es la masa del electrón.
    \part Si en cambio el campo dependiera del tiempo sinusoidalmente,
    como $\bm E(t)=\mathrm{Re} \tilde{\bm E}(t)=\bm E_0 e^{-i\omega
      t}$ el dipolo sería
    $\bm p(t)=\mathrm{Re}\tilde{\bm p}(t)$, con $\tilde{\bm
      p}(t)=alpha(\omega)\tilde{\bm E}(t)$ y la
    polarizabilidad sería
    $$
    \alpha(\omega)=\frac{e^2/m}{\omega_0^2-\omega^2},
    $$
    \part Si además hubiera una fuerza viscosa sobre el electrón
    proporcional a su velocidad, $\bm F_v=-\gamma \bm v$, con $\gamma$
    una constante, la polarizabilidad sería
    $$
    \alpha(\omega)=\frac{e^2/m}{\omega_0^2-\omega^2-i\omega/\tau},
    $$
    donde $\tau=m/\gamma$ es un tiempo característico.
  \end{parts}

  Pista:
  \begin{itemize}
  \item Simplemente resuelve la ecuación de movimiento clásica para el
    electrón.
  \item Es más fácil resolver la ecuación compleja y tomar parte real
    al final del cálculo.
  \end{itemize}
  Nota:
    \begin{itemize}
    \item Curiosamente, el modelo no es tan malo. Para un átomo o
      molécula real, en incluso para un sólido, el resultado es una suma de términos como los
      anteriores, uno para cada posible transición electrónica del
      sistema de muchos electrones, con un peso fraccional conocido
      como {\em fuerza de oscilador}.
    \end{itemize}

  \question Considera un sólido formado por $n$ moléculas por unidad
  de volumen, cada una de ellas no polar, isotrópica y con polarizabilidad $\alpha$.
  \begin{parts}
    \part Argumenta por qué podemos escribir $\bm P=\chi\bm E$ bajo
    ciertas condiciones, donde $\chi=n\alpha$ es la susceptibilidad.
    \part Demuestra que podemos entonces escribir $\bm D=\epsilon \bm E$, donde
    $\epsilon=1+4\pi\chi$ es conocida como permitividad, constante
    dieléctrica o función dieléctrica (pues puede depender de $\omega$).
  \end{parts}

  \question Considera un medio isotrópico, homogéneo e infinito. En el
  origen se coloca una carga (externa) puntual $q$. Demuestra que
  \begin{parts}
    \part el campo eléctrico que produce es
    $$
    \bm E(\bm r)=q\frac{\bm r}{\epsilon r^3}
    $$
    \part y el potencial electríco es
    $$
    \phi(\bm r)=\frac{q}{\epsilon r}.
    $$
  \end{parts}
  Pista:
  \begin{itemize}
  \item Puedes usar la versión integral de la ley de Gauss para
    obtener $\bm D$ y de ahí $\bm E$ y $\phi$.
  \end{itemize}
  Nota:
  \begin{itemize}
  \item En este caso, el resultado es como si la carga $q$ se
    reemplazara por una carga apantallada $q/\epsilon$. ¿Por qué se
    apantalla la carga?
  \end{itemize}
  \question Considera un capacitor formado por dos placas metálicas planas
  paralelas cuadradas de lado $L$ separadas una distancia $d\ll L$. Se
  coloca una carga externa $Q$ en una placa y $-Q$ en la otra.
  \begin{parts}
    \part Despreciando efectos de borde, calcula el campo eléctrico
    entre las placas y afuera de ellas.
  \item Obten el potencial entre las placas.
  \item Encuentra la capacitancia $C=A/4\pi d$, con $A=L^2$ el área de
    las placas.
  \item Repite el cálculo anterior cuando el capacitor está lleno de
    un material con permitividad $\epsilon$.
  \end{parts}
  Pistas:
  \begin{itemize}
  \item Podrías colocar las placas paralelas al plano $xy$, sin pérdida de
    generalidad.
  \item Argumenta que el campo y la densidad de carga son cero en el
    interior de los metales, i.e., la carga se distribuye en la superficie.
  \item Las placas son equipotenciales.
  \item Ignorando los bordes, el potencial sólo depende de $z$.
  \item La solución de la ec. de Poisson entre las placas y afuera es
    entonces lineal en z, y por tanto el campo es constante y apunta
    en la dirección $z$.
  \item Argumenta que el campo afuera de las placas es nulo y obtén el
    campo dentro usando condiciones de contorno sobre $\bm E$.
  \item Intégralo para obtener el potencial $\phi(z)$ y la caida de
    potencial entre la placa positiva y la negativa.
  \item La capacitancia es el cociente entre la carga y la caida de
    potencial.
  \item Repite el cálculo aplicando condiciones de contorno sobre $\bm
    D$.
  \end{itemize}
  Nota:
  \begin{itemize}
  \item La capacitancia se multiplica por $\epsilon$ debido al relleno
    del capacitor.
  \end{itemize}
  \question Un cilindro metálico con radio exterior $a$ se coloca
  dentro de un cilindro metálico hueco coaxial con radio interior $b>a$. Ambos
  tienen la misma altura $h>>b$. El espacio entre ellos se llena de un
  dieléctrico con permitividad $\epsilon$. Encuentra la capacitancia del
  sistema.

  Pista:
  \begin{itemize}
  \item Coloca cargas $\pm Q$ en los dos cilindros.
  \item Las dos superficies metálicas son equipotenciales.
  \item La carga se distribuye entonces uniformemente en sus
    superficies.
  \item El campo es radial, no depende ni de $z$ y su componente
    radial no depende de $\theta$.
  \item Usa la ley de Gauss en su forma integral para obtener la
    comonente radial de $D$, luego de $E$, y luego intégrala para
    obtener el potencial.
  \item La capacitancia es la carga dividida entre la caida de
    potencial.
  \end{itemize}

  \question Considera un dieléctrico semiinfinito con permitividad
  $\epsilon_1$ ocupando la región $z>0$ y otro ocupando la región
  $z<0$. Se coloca una carga puntual $q$ sobre el eje $z$ a una
  distancia $d>0$ del origen.
  \begin{parts}
    \part Demuestra que el potencial que produce es
    $$
    \phi(\bm r)= \frac{q/\epsilon_1}{(x^2+y^2+(z-d)^2)^{1/2}}
    +\frac{q'/\epsilon_1}{(x^2+y^2+(z+d)^2)^{1/2}}
    $$
    en $z>0$ y como
    $$
    \phi(\bm r)= \frac{q''/\epsilon_2}{(x^2+y^2+(z-d)^2)^{1/2}}
    $$
    en $z<0$, donde
    $$
    q'=\frac{\epsilon_1-\epsilon_2}{\epsilon_1+\epsilon_2}q
    $$
    y
    $$
    q''=\frac{2\epsilon_2}{\epsilon_1+\epsilon_2}q
    $$
  \end{parts}
  Pistas:
  \begin{itemize}
  \item Usa el método de imágenes. Argumenta que añadir cargas
    ficticias en $z<0$ no afecta las ecuaciones de campo en $z>0$ y
    añadir cargas ficticias en $z>0$ no afecta las ecuaciones de campo
    en $z<0$.
  \item Propón como {\em anzatz} una solución en la región $z>0$ dada
    por el potencial que produciría la carga original $q$ en el eje z
    en $z=d$ y una carga ficticia, {\em la carga imágen} colocada
    también en el eje $z$ (dónde si no) y a una distancia $d'$ hacia
    abajo del origen, ambas en un medio ficticio homogeneo con
    permeabilidad $\epsilon_1$. La carga y la permeabilidad de este
    sistema ficticio coinciden con las reales en la región $z<0$,
    aunque no en la región $z<0$.
  \item Propón como {\em anzatz} una solución en la región $z<0$ dada
    por el potencial que produciría una carga ficticia $q''$ colocada
    en el eje $z$ a una distancia $d''$ arriba del origen dentro de un
    medio homogéneo con permeabilidad $\epsilon_2$. La carga y la
    permeabilidad coinciden con las reales en la región $z<0$, pero no
    en la región $z>0$.
  \item Impón condiciones de contorno sobre el potencial en la
    interfaz $z=0$, i.e., el potencial justo arriba debe ser igual al
    potencial justo abajo, para toda $x$ y $y$.
  \item Impón condiciones de contorno para las comonentes normales a
    la superficie del desplazamiento.
  \item Demuestra que eligiendo juiciosamente $d'$ y $d''$, las
    condiciones de contorno se cumplirían para toda $x$ y $y$ con
    cumplirse en un solo punto.
  \item Escribe entonces ambas ecuaciones de contorno como dos
    ecuaciones lineales con dos incógnitas, $q'$ y $q''$ y
    resuélvelas.
  \end{itemize}
\end{questions}
\end{document}
