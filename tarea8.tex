\documentclass{exam}
\usepackage{espanhol}
\usepackage{bm}
\usepackage{amsmath}
\usepackage{comment}
\usepackage{braket}
%\usepackage{showkeys}
% \usepackage{pstricks}
\decimalpoint
\pointname{\ puntos}
\cfoot[]{\thepage}
\newenvironment{pistas}{\par\noindent Pistas:\begin{enumerate}} {\end{enumerate}}
\newenvironment{notas}{\par\noindent Notas:\begin{enumerate}} {\end{enumerate}}

\begin{document}
\begin{center}
  \bf\large Tarea 8\\
  Propedéutico Electrodinámica\\
  2020-05-10\\[20pt]
\end{center}
\hbox to \textwidth{Nombre: \enspace\hrulefill}
Entregar el viernes 2020-05-15.

\begin{questions}
\question\label{a} Deduce las ecuaciones de Maxwell {\em
    macroscópicas} en materiales
  polarizables y magnetizables.
  \begin{pistas}
  \item Parte de las ecuaciones inhomogéneas de Maxwell en términos de
    $\bm E$ y $\bm B$ y las fuentes totales $\rho^{\text{tot}}$ y
    $\bm j^{\text{tot}}$ (las llamadas, ecuaciones de Maxwell {\em microscópicas}).
  \item Escribe separa las fuentes en fuentes externas (no
    pertenecientes al material) e internas (pertenecientes al
    material), $\rho^{\text{tot}}=\rho^{\text{ext}}+\rho^{\text{int}}$,
    $\bm j^{\text{tot}}=\bm j^{\text{ext}}+\bm j^{\text{int}}$.
  \item Escribe las fuentes internas en términos de la polarización y
    la magnetización, $$\rho^{\text{int}}=-\nabla\cdot\bm P$$ y $$\bm
    j^{\text{int}}=\frac{\partial}{\partial t}\bm P+c\nabla\times\bm
    M$$.
  \item Identifica en las ecuaciones resultantes a $\bm D$ y $\bm H$.
  \item Las ecuaciones homogéneas no se ven modificadas por la
    presencia de materiales.
  \item La solución es $\nabla\cdot\bm D=4\pi\rho^{\text{ext}}$,
      $\nabla\times\bm H=(4\pi/c)\bm j^{\text{ext}}+(1/c)\partial\bm
      D/\partial t$, $\nabla\cdot\bm B=\bm 0$, $\nabla\times\bm
      E=-(1/c)\partial\bm B/\partial t$.
  \end{pistas}
  \begin{notas}
  \item Como las ecuaciones homogéneas no se modifican, la existencia
    de potenciales electromagnéticos $\phi$, $\bm A$ y su relación con
    $\bm E$ y $\bm B$ tampoco.
  \end{notas}

\question\label{b} Encuentra las condiciones de contorno que cumplen los campos
  electromagnéticos $\bm E$, $\bm B$, $\bm D$ y $\bm H$, así como $\bm
  P$ y $\bm M$ en una superficie cualquiera.
  \begin{pistas}
  \item Este problema no es más que un repaso de problemas resueltos
    en las tareas 3 y 5. Simplemente, hemos de verificar que los
    resultados son válidos en el caso general y no solo en el estático
    o el estacionario.
  \item Se llega a la solución integrando las ecuaciones para las
    divergencias en un pequeño cilindro que atrapa a la superficie
    entre sus tapas o integrando las ecuaciones para los rotacionales
    sobre pequeños rectángulos que atrapan la superficie entre dos de
    sus lados.
  \item La solución es $\Delta B_\perp=0$, $\Delta \bm E_\parallel=0$,
    $\Delta\bm D_\perp=4\pi\sigma^{\text{ext}}$, $\Delta
    \bm H_\|=(4\pi/c)\bm K^{\text{ext}}\times\hat{\bm n}$, $\Delta
    P_\perp=-\sigma^{\text{int}}$, $\Delta\bm M_\|=\bm
    K^{\text{int}}\times\hat{\bm n}/c$. En ausencia de una densidad de
    carga externa singulare sobre la superficie, $\sigma^{\text{ext}}=0$, la
    componente normal a la superficie del desplazamiento es continua
    en la superficie, $\Delta
    D_\perp=0$, y en ausencia de una densidad corriente externa
    singular en la superficie, $\bm K^{\text{ext}}=0$, la proyección
    tangencial a la superficie del campo magnético es continua en la superficie,
    $\Delta\bm H_\|=0$.
  \end{pistas}
\question\label{c} Transforma las ecuaciones de Maxwell sin fuentes en ecuaciones
  algebraicas para el caso de una onda plana monocromática con vector de onda $\bm
  k$ y frecuencia $\omega$ propagandose en un material polarizable y
  magnetizable
  \begin{pistas}
  \item Considera campos de la forma $\bm F(\bm r,t)=\bm F_0 e^{i(\bm
      k\cdot\bm r-\omega t)}$ donde $\bm F$ representa a cualquiera de
    $\bm E$,   $\bm D$,   $\bm B$ y  $\bm H$, y donde se sobreentiende
    que debemos tomar la parte real al terminar el cálculo.
  \item Argumenta que podemos hacer los siguientes reemplazos:
    $\nabla\to i\bm k$, $\partial/\partial t\to -i\omega$.
  \item Concluye que
    $\bm k\cdot\bm B=0$, $\bm k\times\bm
    E=(\omega/c)\bm B$, y $\bm k\cdot\bm D=0$ y $\bm k\times\bm H=-(\omega/c)\bm D$.
  \end{pistas}
  \begin{notas}
  \item $\bm k$ es ortogonal a $\bm D$ y $\bm B$, $\bm B$ es ortogonal
    a $\bm E$ y $\bm D$ es ortogonal a $\bm H$
  \item Nota que $\bm k$, $\bm E$ y $\bm B$ forman una terna ordenada
    derecha.
  \item Nota que $\bm k$, $\bm D$ y $\bm H$ también.
  \item En materiales isotrópicos $\bm D\|\bm E$ y $\bm B\|\bm H$ y
    todos los vectores de las ternas previas son mutuamente ortogonales.
  \end{notas}
\question\label{d} Encuentra la relación de dispersión para una onda
  electromagnética propagándose libremente en un medio polarizable y magnetizable
  lineal e isotrópico, caracterizado por una permitividad
 $\epsilon$ y una permeabilidad $\mu$.
 \begin{pistas}
     \item Sustituye las ecuaciones materiales $\bm D=\epsilon\bm E$ y
       $\bm B=\mu \bm H$ en las ecuaciones {\em algebráicas de Maxwell} del
       problema anterior.
     \item Toma el producto vectorial con $\bm k$ de las ecuaciones
       que incluyen productos vectoriales.
     \item Usa la identidad $\bm k\times(\bm k\times\ldots)=\bm k\bm
       k.\ldots-k^2\ldots$
     \item Usa las ecuaciones de Maxwell para el producto punto para
       eliminar los términos de la forma $\bm k\bm k\cdot\ldots$
     \item Llega a $k^2\bm E=\epsilon\mu(\omega^2/c^2)\bm E$ y  $k^2\bm
       H=\epsilon\mu(\omega^2/c^2)\bm H$.
     \item Por tanto, $k=n\omega/c$ donde $n=\sqrt{\epsilon\mu}$ es el
       índice de refracción.
 \end{pistas}
 \begin{notas}
 \item Nota que el campo electromagnético es {\em transversal}, i.e.,
   $\bm E$, $\bm B$, $\bm D$ y $\bm H$ son perpendiculares al vector
   de onda $\bm k$, i.e., a la dirección
   de propagación.
 \item La velocidad de propagación de la onda es
   $v_f=\omega/k=c/n$, donde el subíndice $f$ denota que ésta es la
   {\em velocidad de fase}.
 \item Típicamente, la permeabilidad a altas frecuencias es $\mu=1$,
   por lo que $n=\sqrt\epsilon$.
 \item Si $\epsilon\mu<0$, $n$ es imaginario, el vector de onda se
   vuelve complejo y la onda no se propaga en el medio, sino que decae
   exponencialmente. El medio se vuelve opaco. El medio también es
   opaco si $\epsilon$ o $\mu$ no son reales, i.e., si hay absorción.
 \item Para $n$ real, $B=nE$.
 \end{notas}
\question\label{e} Encuentra el teorema de Poynting en el interior de un
  material lineal isotrópico caracterizado por $\epsilon$ y $\mu$.
  \begin{pistas}
  \item El teorema de Poynting de la tarea 7 es correcto, pero
    describe el intercambio de energía con todas las cargas.
  \item La potencia que pierda el campo por unidad de volumen debido a
    su interacción con las cargas {\em externas}, en lugar de su
    interacción con todas las cargas, es $-\bm j^{\text{ext}}\cdot\bm
    E$.
  \item En analogía con la tarea 7, la ley de conservación de la
    energía es $\partial u/\partial t+\nabla\cdot\bm S=-\bm
    j^{\text{ext}}\cdot\bm E$.
  \item Aquí, $u$ y $\bm S$ incluyen la energía asociada a las cargas
    internas y su movimiento.
  \item Sustituyendo $\bm j^{\text{ext}}$ de la ley de Ampère-Maxwell
    en materiales y haciendo las mismas manipulaciones que en la tarea
    7, obtén $u=(1/8\pi)(\bm E\cdot\bm D+\bm H\cdot\bm B)$ y $\bm
    S=(c/4\pi)\bm E\times\bm H$.
  \end{pistas}
\question\label{f} Una onda electromagnética plana monocromática, como
  en el problema \ref{c}, con frecuencia $\omega$, vector de onda $\bm
  k$, campo eléctrico $\bm E$ con amplitud $\bm E_0$, se propaga en un
  medio transparente caracterizado por $\epsilon$ y $\mu$. Calcula su flujo y su
  densidad de energía.
  \begin{pistas}
  \item Dado $\bm E_0$ podemos calcular los campos
    $\bm E$, $\bm D$, $\bm B$ y $\bm H$ usando las ecuaciones
    materiales y las ecuaciones de Maxwell para ondas planas (problema
    \ref{c}).
  \item No podemos directamente reemplazar campos complejos de la
    forma $\bm F=\bm F_0 e^{i(\bm k\cdot\bm r-\omega t)}$ en las
    expresiones para $u$ y $\bm S$ pues antes es necesario tomar las
    partes reales, como fue discutido en el problema 24 de la tarea 7.
  \item Podemos, siguiendo dicho problema, emplear las expresiones
    para el promedio temporal de la densidad de energía y del flujo de
    energía $\braket{u}=\text Re({\bm E\cdot\bm D^*+\bm H\cdot\bm
      B^*})/16\pi$ y $\braket{\bm S}=(c/8\pi)\text{Re}(\bm E\times
    H^*)$.
  \item El resultado es $\braket{u}=\epsilon |E|^2/(8\pi)$ y
    $\braket{\bm S}=(c/8\pi) \sqrt{\epsilon/\mu}|E|^2\hat{\bm k}$ con
    $\hat{\bm k}$ un vector unitario en la dirección de propagación.
  \end{pistas}
  \begin{notas}
  \item Podemos escribir $\bm S=u v_f\hat{\bm k}$, i.e., el flujo es
    densidad por velocidad, como el gasto en un fluido o la corriente
    eléctrica.  En este caso, la velocidad es la velocidad de fase y
    la densidad es la densidad de energía.
  \item Se define la intensidad $I$ de una onda mediante $\braket{\bm
      S}=I\hat{\bm k}$.
  \item Podemos reescribir el resultado como
    $I=(c/8\pi)\sqrt{\epsilon/\mu}|E|^2$.
  \item Hay sutilezas que considerar cuando $\epsilon$ y $\mu$
    dependen de la frecuencia.
  \item Un cálculo más cuidadoso emplearía un grupo de ondas, no ondas
    planas, para las cuales $\nabla\cdot\braket{\bm S}=0$ y $\partial
    \braket{u}/\partial t=0$.
  \end{notas}
\question\label{g} Considera un material $a$ semiinfinito ocupando la región
  $z<0$ en contacto con un material $b$ semiinfinito ocupando la
  región $z>0$. Los materiales tienen permitividades $\epsilon_a$ y
  $\epsilon_b$ y son no magnéticos, $\mu_a=\mu_b=1$. Una onda plana
  monocromática con frecuencia $\omega$ linealmente polarizada incide
  de abajo hacia arriba sobre la interfaz $z=0$ moviéndose en la
  dirección $z$ . Calcula la reflectancia y la transmitancia.
  \begin{pistas}
  \item La reflectancia $R$ y transmitancia $T$ se definen como los
    cocientes
    entre las componentes normales a la superficie de los flujos de
    energía en las ondas reflejada y transmitida. En este caso,
    $R\equiv -\braket{S_r^z}/\braket{S_i^z}$ y
    $T=\braket{S_t^z}/\braket{S_i^z}$, donde $i$, $r$ y $t$
    denotan a las ondas incidente, reflejada y transmitida,
    respectivamente.
  \item Argumenta por qué la onda reflejada y la transmitida tienen la
    misma frecuencia $\omega$ que la onda incidente y se propagan a lo
    largo del eje $z$.
  \item Sin pérdida de generalidad, podemos tomar al campo eléctrico
    de la onda incidente a lo largo del eje $x$.
  \item Argumenta por qué en este caso, los campos eléctricos de todas
    las ondas apuntan a lo largo de $x$ y todos los campos magnéticos
    van a lo largo de $y$.
  \item Toma $E_i^x(z,t)=E_0 e^{i(k_az-\omega t)}$.
  \item ¿Cuánto vale $k_a$?
  \item ¿Cuanto vale $B_i^y(z,t)$?
  \item ¿Cuánto vale $\braket{S_i^z}$?
  \item Escribe el campo eléctrico reflejado como $E_r^x(z,t)=r E_0
    e^{-i(k_az+\omega t)}$.
  \item $r$ es la {\em amplitud de reflexión} por ser determinada.
  \item ¿Cuánto vale $B_r^y(z,t)$? (Cuidado con los signos)
  \item Escribe el campo eléctrico transmitido como $E_t^x(z,t)=t E_0
    e^{i(k_bz-\omega t)}$.
  \item $t$ es la {\em apmlitud de transmisión} por ser determinada.
  \item ¿Cuánto vale $k_b$?
  \item ¿Cuánto vale $B_t^y(z,t)$?
  \item Plantea las condiciones de contorno independientes que obedece
    el campo electromagnético en $z=0$. Verifica que hay cuatro
    condiciones de contorno, pero en este caso sólo dos son independientes.
  \item Resuelve las ecuaciones subsecuentes para obtener las dos incógnitas $r$ y $t$.
  \item Encuentra $\braket{S_r^z}$ y  $\braket{S_t^z}$.
  \item Calcula $R$ y $T$.
  \item El resultado es $R=|(n_a-n_b)/(n_a+n_b)|^2$ y $T=4n_an_b/|n_a+n_b|^2$.
  \end{pistas}

  \begin{notas}
  \item La polarización es irrelevante. Escogí polarización lineal por
    simplicidad.
  \item Se puede generalizar el problema anterior a otros ángulos de
    incidencia.
  \item Para ello, primero hay que establecer la ley de la reflexión y
    la ley de Snell.
  \item También hay que separar los casos de polarización transversal
    eléctrica TE y transversal magnética TM.
  \end{notas}
\question\label{h} Considera dos ondas planas monocromáticas con la
  misma polarización lineal que se propagan en un medio caracterizado
  por $\epsilon$ y $\mu$ con la misma amplitud, frecuencia y dirección
  de propagación, pero difiriendo por una fase $\psi$. Calcula la
  intensidad $I$ de la onda resultante.
  \begin{pistas}
  \item Calcula el campo total y úsalo para calcular la intensidad.
  \end{pistas}
  \begin{notas}
  \item Dependiendo de su fase relativa, la intensidad de la suma de
    dos campos iguales copropaantes puede variar desde 4 veces la de
    una onda sola hasta 0.
  \item Si la fase es 0 tenemos {\em interferencia constructiva}.
  \item Si la fase es $\pi$ tenemos {\em interferencia destructuva}.
  \end{notas}

\question\label{i} Considerea dos ondas planas monocromáticas de
  frecuencia $\omega$ con polarización lineal a lo largo de $x$ con la
  misma amplitud $E_0$ y que se propagan una en la dirección $z$ y otra en
  la dirección $-z$ en un medio no magnético con permitividad
  $\epsilon$ y permeabilidad $\mu$.
  \begin{parts}
  \part Calcula el campo eléctrico $E_x(z,t)$.
  \part Calcula el campo magnético $B_y(z,t)$.
  \part Calcula la densidad de energía $\braket{u(z,t)}$.
  \part Calcula el flujo de energía $\braket{S_z(z,t)}$.
  \end{parts}
  \begin{notas}
  \item Esta es una onda estacionaria.
  \item La distancia entre nodos sucesivos o entre antinodos sucesivos
    es media longitud de onda.
  \end{notas}
\question\label{j} Considera dos ondas planas monocromáticas de
  frecuencias $\omega_1=\omega_0+\delta\omega$ y
  $\omega_2=\omega_0-\delta\omega$ moviéndose en la dirección $z$ con
  polarización lineal a lo largo de $x$ y con amplitudes $E_{10}$ y
  $E_{20}$ en un medio con permitividad $\epsilon(\omega)$ {\em
    dependiente} de la frecuencia y con permeabilidad
  $\mu=1$. Suponiendo que $\delta\omega$ es muy pequeño:
  \begin{parts}
  \part Calcula el campo resultante $E_x(z,t)$.
  \part Demuestra que $E_x(z,t)$ se puede escribir como el producto de
    una onda portadora monocromática de frecuencia $\omega_0$
    multiplicada por una envolvente.
  \part ¿Cuál es la velocidad de la onda portadora?
  \part ¿De qué frecuencia es la envolvente?
  \part ¿Cual es la velocidad de la envolvente?
  \end{parts}
  \begin{pistas}
  \item Aproxima
    $\epsilon(\omega_0\pm\delta\omega)\approx\epsilon(\omega_0)\pm\delta\omega
    d\epsilon/d\omega$.
  \item Realiza las aproximaciones correspondientes para calcular los
    números de onda $k(\omega_0\pm\delta\omega)$.
  \end{pistas}
  \begin{notas}
  \item La velocidad de la envolvente se conoce como {\em velocidad de
      grupo} $v_g$.
  \item Cuando $\epsilon$ no depende de $\omega$, la velocidad de
    grupo es igual a la de fase.
  \item Si $\epsilon$ es creciente, la velocidad de grupo es menor que
    la de fase. Este caso se conoce como {\em dispersión normal}.
  \item En el caso opuesto, tenemos {\em dispersión anómala}.
  \end{notas}
\question\label{k} Una onda plana monocromática de frecuencia $\omega$ se
  propaga en el espacio vacío e incide normalmente sobre una pantalla opaca
  delgada en la que se han cortado dos ventanas delgadas largas idénticas
  separadas una distancia $d$. A una distancia $L\gg d$ se coloca una
  pantalla.
  \begin{parts}
  \part Calcula la intensidad normalizada
    $I(\theta)/I(0)$ como función del ángulo $\theta$ respecto a la
    dirección de propagación original (para ángulos pequeños).
  \part ¿Cual es la distancia angular $\Delta\theta$ entre máximos de
    intensidad?
  \end{parts}
  \begin{pistas}
  \item De acuerdo al principio de Kirchoff, cada punto en un frente
    de onda se comporta como un emisor de ondículas secundarias esféricas que se
    propagan en todas las direcciones.
  \item Similarmente, una línea recta en un frente de onda con
    amplitud constante a lo largo de la línea se comporta como la
    fuente de ondículas cilíndricas que se propagan en todas las
    direcciones normales a la línea. Proyectando sobre un plano normal
    a la línea, tendríamos ondas circulares.
  \item El decaimiento de la onda con la distancia se puede ignorar,
    pues a largas distancias es más rápido su cambio de fase.
  \item Además, se pide el resultado {\em normalizado} a la intensidad
    central.
  \item Entonces, hay que sumar dos campos cuya fase difiere por haber
    recorrido distancias distintas, y de ahí obtener la intensidad.
  \end{pistas}
  \begin{notas}
  \item El fenómeno en que un haz luminoso se separa en varios haces
    con distintas direcciones en al pasar por obstáculos se conoce
    como {\em difracción}.
  \item La difracción por una {\em doble rejilla} fue un experimento
    clave para establecer el caracter ondulatorio de la luz.
  \end{notas}
\question\label{l} Una onda plana monocromática de frecuencia $\omega$ se
  propaga en el espacio vacío e incide normalmente sobre una pantalla opaca
  delgada en la que se ha cortado una ventana larga de ancho $d$. A
  una distancia $L\gg d$ se coloca una
  pantalla. Calcula la intensidad la intensidad normalizada
  $I(\theta)/I(0)$ como función del ángulo $\theta$ respecto a la
  dirección de propagación original (para ángulos pequeños).
  \begin{pistas}
  \item Divide la ventana rectangular en un gran número de rectángulos
    largos muy delgados, de ancho $\delta x$.
  \item Aplica el principio de Kirchoff a cada rectángulo.
  \item Suma los campos resultantes.
  \item Convierte la suma en una integral tomando el límite $\delta
    x\to 0$.
  \end{pistas}
  \begin{notas}
  \item Así como hay un principio de incertidumbre $\Delta p\Delta
    x>\hbar$ en mecánica cuántica, hay análogos clásicos que
    relacionan, por ejemplo $\Delta\omega$ con $\Delta t$ para un
    paquete de onda. Este problema ilustra un principio de
    incertidumbre que relaciona $\Delta x$ con $\Delta
    \theta$. Mientras mejor sabemos por dónde está confinada una onda
    (en el interior de la ventana), menos sabemos qué dirección
    tiene. Sólo las ondas planas tienen una dirección bien
    definida. ¿Puedes estimar en este problema $\Delta x\Delta\theta$?
  \end{notas}
\question \label{m} Demuestra que la ecuación de onda para un campo
  $\phi(\bm r,t)$ con fuentes dadas $f(\bm r,t)$,
  $(\nabla^2-(1/c^2)\partial^2/\partial t^2)\phi(\bm r, t)=-4\pi f(\bm
  r,t)$, se convierte en la {\em ecuación de Helmholtz}
  $(\nabla^2+k^2)\phi_\omega(\bm r)=-4\pi f_\omega(\bm r)$ cuando las
  fuentes son monocromáticas con frecuencia $\omega$, donde $k=\omega/c$.
  \begin{pistas}
  \item Sustituye $f(\bm r,t)=f_\omega(\bm r) e^{-i\omega t}$, propón
    el {\em ansatz} estacionario $\phi(\bm r,t)=\phi_\omega(\bm r)
    e^{-i\omega t}$ y evalua las derivadas temporales.
  \end{pistas}
\question \label{n} Demuestra que una solución particular de la
  ecuación de Helmholtz con fuentes es
  $$
  \phi_\omega(\bm r)=\int d^3r'\, G_\omega(\bm r-\bm r') f_\omega(\bm r'),
  $$
  donde $G_\omega(\bm r)$ obedece la ecuación de Helmholtz con una
  fuente puntual unitaria colocada en el origen,
  $$
  (\nabla^2+k^2) G_\omega(\bm r)=-4\pi\delta(\bm r),
  $$
  $k=\omega/c$ y $\delta(\bm r)$ es la delta de Dirac.
  \begin{pistas}
  \item Por simplicidad, tomemos a la delta como una {\em función} que
    vale cero en todo el espacio excepto en el origen, y cuya integral
    en cualquier región que incluya al origen es 1,
    $\int_Vd^3r\delta(\bm r)=1$ si $\bm 0\in V$.
  \item Si no estás familiarizado con ella, considera una función que
    vale cero en todo el espacio excepto en una pequeña región de
    tamaño despreciable sobre la cual la integral vale 1.
  \item En todo caso, si la región es suficientemente chica,
    $\int d^3r'\,\delta(\bm r-\bm r')f_\omega(\bm r')=f_\omega(\bm r)$. De hecho,
    ésta esta está relacionada con una mejor {\em definición} de la
    delta de Dirac, cono {\em funcional}.
  \item Aplica el operador de Helmholtz a la solución propuesta e
    intercambialo con la integral para encontrar
    $$
    (\nabla^2+k^2)\int d^3r'\, G_\omega(\bm r-\bm r') f_\omega(\bm
    r')=-4\pi\int d^3r'\, \delta(\bm r-\bm r') f_\omega(\bm r')=-4\pi f_\omega(\bm r).
    $$

  \end{pistas}
\question \label{o} Demuestra que
  $$
  G_\omega(\bm r)=\frac{e^{\pm ikr}}{r}.
  $$
  es una función de Green isotrópica saliente para la ecuación de Helmholtz.
  \begin{pistas}
  \item Demuestra que para $r\ne 0$ cumple la ecuación de Helmholtz.
  \item Argumenta por qué a distancias chicas $r\ll 1/k$ se puede
    aproximar por $1/r$, la función de Green electrostática, el
    potencial coulombiano.
  \item Argumenta por qué a distancias chicas se puede ignorar el
    factor $k^2$ en la ecuación de Helmholtz.
  \item Por tanto $(\nabla^2+k^2)(e^{ikr}/r)=\nabla^2 1/r$ a
    distancias chicas, y eso es como la densidad de carga
    correspondiente a una carga puntual unitaria en el origen, i.e.,
    una delta de Dirac.
  \item Al introducir la dependencia temporal, multiplicando por
    $e^{-i\omega t}$, y alejarnos a grandes distancias $r\gg 1/k$ en
    cualquier dirección, obtenemos una onda que se aleja del
    origen. Por eso ésta solución se llama saliente.
  \item La solución {\em entrante} es $e^{-ikr}/r$.
  \end{pistas}
\question \label{p} Considera una distribución de corriente $\bm
  j(\bm r,t)=\bm j(\bm r)e^{-i\omega t}$ que oscila con frecuencia
  $\omega$ bien definida. Demuestra que los potenciales
  electromagnéticos en la {\em norma de Lorentz} están dados por
  $\phi(\bm r,t)=\phi(\bm r)e^{-i\omega t}$ y   $\bm A(\bm r,t)=\bm
  A(\bm r)e^{-i\omega t}$ con
  $$
  \phi(\bm r)=\int d^3r'\, \frac{e^{ik|\bm r-\bm r'|}}{|\bm r-\bm r'|}
  \rho(\bm r')
  $$
  y
  $$
  \bm A(\bm r)=\frac{1}{c}\int d^3r'\, \frac{e^{ik|\bm r-\bm r'|}}{|\bm r-\bm r'|}
  \bm j(\bm r')
  $$
  \begin{pistas}
  \item En la norma de Coulomb $\phi$ y $\bm A$ obedecen la ecuación
    de onda con fuentes $\rho$ y $\bm j/c$ respectivamente.
  \item Para fuentes monocromáticas, podemos reemplazarla por la
    ecuación de Helmholtz.
  \item Podemos usar la función de Green del problema \ref{o}.
  \end{pistas}
\question \label{q} Considera un sistema como el del inciso \ref{p},
  pero en el cual las fuentes están totalmente contenidas en una
  pequeña región de radio $R\ll\lambda$, donde $\lambda=2\pi/k$ es la
  longitud de onda y $k=\omega/c$. Demuestra que a distancias muy
  grandes $r\ll\lambda$ el potencial vectorial está dado
  aproximadamente por
  $$
  \bm A(\bm r)=\frac{e^{ikr}}{rc}\int d^3r'\, \bm j(\bm r').
  $$
  \begin{pistas}
  \item Haz una expansión de Taylor en $r'$ y conserva el término más
    grande cuando $r'\ll r$ y $kr'\ll 1$.
  \end{pistas}
  \begin{notas}
  \item Este es el término dominante del potencial en la {\em zona de radiación}
    ($r\gg\lambda$) en la {\em aproximación de longitud de onda larga}
    $r'<R\ll\lambda$.
  \end{notas}
\question \label{r} Evalúa la integral del problema \ref{q} en
  términos del momento dipolar eléctrico del sistema.
  \begin{pistas}
  \item Queremos evaluar $\int d^3r'\, \bm j(\bm r')=\int d^3r\, \bm j(\bm r)$.
  \item Escribe cada componente del integrando $j_i(\bm r)$ como
    $$
    j_i(\bm r)=j_k(\bm r)\delta_{ik}=j_k(\bm r)\frac{\partial
      r_i}{\partial r_k}
    $$
    donde $\delta_{ik}$ es la {\em delta de Kronecker} y empleamos la
    notación de Einstein (suma implícita sobre índices repetidos, en
    este caso $k$).
  \item Manipula las derivadas para obtener
    $$
    \int d^3r\, j_i(\bm r)= \int d^3r\, \nabla\cdot(\bm j(\bm r)
    r_i)-\int d^3r\, (\nabla\cdot\bm j(\bm r)) r_i.
    $$
  \item Usa el teorema de Gauss y el hecho de que la carga está cerca
    del origen para argumentar que la primera integral vale cero.
  \item En la segunda integral usa la ecuación de continuidad
    $\nabla\cdot\bm j(\bm r,t)=-\partial\rho(\bm r,t)/\partial
    t=i\omega\rho(\bm r,t)$.
  \item Finalmente $\int d^3r\, j_i(\bm r)= -i\omega p_i$, donde
    $\bm p=\int d^3r\, \rho(\bm r)\bm r$ y $\bm p(t)=\bm p e^{-i\omega
      t}$ es el momento dipolar de la distribución de cargas.
  \end{pistas}
\question \label{s} Calcula la contribución dipolar al potencial
  vectorial $\bm A$, el campo magnético $\bm B$ y el campo eléctrico
  $\bm E$ en la zona de radiación en la aproximación de longitud de
  onda larga.
  \begin{pistas}
  \item Sustituye la solución del problema \ref{r} en la del problema
    \ref{q} para obtener $\bm A(\bm r)$.
  \item El campo magnético $\bm B(\bm r)$ se puede obtener calculando
    $\nabla\times\bm A$.
  \item Argumenta por qué en la zona de radiación podemos aproximar
    $\nabla\approx ik\hat{\bm n}$, donde $\hat{\bm n}=\bm r/r$ es un
    vector unitario en la dirección de observación.
  \item Utiliza esta aproximación para evaluar el campo magnético $\bm
    B(\bm r)$.
  \item El campo eléctrico se puede obtener a partir de la Ley de
    Ampère-Maxwell $\nabla\times\bm B(\bm r,t)=(1/c)(\partial/\partial
    t)\bm E(\bm r,t)=-ik\bm E(\bm r,t)$.
  \item Aproxima el rotacional en la zona de radiación.
  \item Los resultados son $\bm A(\bm r)=-ik\bm p e^{ikr}/r$, $\bm
    B(\bm r)=k^2 \hat{\bm n}\times\bm p e^{ikr}/r$ y $E(\bm r)=-k^2
    \hat{\bm n}\times(\hat{\bm n}\times\bm p) e^{ikr}/r$.
  \end{pistas}
\question \label{t} Calcula la potencia radiada por unidad de ángulo
  sólido en el caso de la radiación dipolar.
  \begin{pistas}
  \item El flujo de energía está dado por el vector de Poynting
    $\braket{\bm S}=\frac{c}{8\pi}\text Re\, \bm E\times \bm B^*$.
  \item Sustituyendo los resultados para radiación dipolar se obtiene
    $\braket{\bm S}=\frac{c}{8\pi}|\hat{\bm n}\times\bm p|^2\hat{\bm
      n}/r^2$.
  \item Considera una esfera de radio $r\ll\lambda$ centrada en el
    origen. Considera un cono angosto con vértice en el origen y que
    cubre un ángulo sólido $d\Omega$. La intersección del cono con la
    esfera es un área $da=r^2d\Omega$.
  \item La potencia que atraviesa dicha área es $d\mathcal P=
    \bm S\cdot\hat{\bm n}da=\bm S\cdot\hat{\bm n}r^s d\Omega$.
  \item De aquí, obtén la potencia por unidad de ángulo sólido.
  \item Sustituye el vector de Poynting calculado arriba.
  \item El resultado es $d\mathcal P/d\Omega=(ck^4/8\pi) |\hat{\bm n}\times\bm p|^2$
  \end{pistas}
  \begin{notas}
  \item Nota que esta potencia no depende de la distancia en la zona
    de radiación y que siempre es positiva.
  \item Esto significa que un dipolo oscilante radía y pierde energía
    irreversiblemente.
  \item Las oscilaciones dipolares son entonces una fuente de energía luminosa.
  \end{notas}
\question \label{u} Calcula y dibuja el patrón de radiación de energía
  para un momento dipolar
  \begin{parts}
  \part que oscila con frecuencia $\omega$ a lo largo del eje $z$;
  \part que gira en el plano $xy$ siguiendo un círculo con radio
    $p_0$.
  \end{parts}
  \begin{pistas}
  \item En ambos casos escribe el momento dipolar $\bm p(t)$ como
    $\text Re\, \bm p_0 e^{-i\omega t}$, encuentra la amplitud
    (compleja) $\bm p_0$ de la oscilación y sustitúyela en el
    resultado de los problemas previos.
  \end{pistas}
\question \label{v} Calcula la potencia total radiada por un dipolo
  que oscila con frecuencia $\omega$.
  \begin{pistas}
  \item Integra $d\mathcal P/d\Omega$ sobre el ángulo sólido.
  \end{pistas}
  \begin{notas}
  \item El resultado, $\mathcal P=c k^4|\bm p|^2/3$ se conoce como la
    {\em fórmula de Larmor}.
  \end{notas}
\question \label{w}Considera un modelo clásico para el átomo de
  hidrógeno consistente en un electrón que gira alrededor de un protón
  en una órbita circular de radio $a_B$ (el radio de Bohr). Calcula
  qué la energía que pierde por radiación durante cada vuelta.
  \begin{pistas}
  \item Calcula clásicamente el momento dipolar instantáneo, la
    amplitud del dipolo giratorio, la frecuencia de giro y sustituye en la
    fórmula de Larmor.
  \end{pistas}
  \begin{notas}
  \item Este problema demuestra que un átomo clásico sería inestable
    frente a pérdidas de energía radiativas.
  \item Un cálculo completo muestra que el electrón colapsaría al
    nucleo en un tiempo finito pequeñito. Calcúlalo o estímalo
  \item La predicción clásica de que la materia colapsaría rápidamente
    por las pérdidas radiativas de energía es uno de los motivos que
    llevaron al surgimiento de la mecánica cuántica.
  \end{notas}
\question\label{w} Considera una molécula con una polarizabilidad
  $\alpha$ iluminada por una onda plana monocromática con frecuencia
  $\omega$. Calcula su sección transversal de esparcimiento (dispersión).
  \begin{pistas}
  \item El campo eléctrico oscilante $\bm E$ de la onda plana induce
    un momento dipolar oscilante $\bm p=\alpha\bm E$.
  \item Por tanto, la molécula radiaría energía electromagnética en
    todas direcciones, esparciendo una potencia dada por la fórmula de
    Larmor.
  \item La sección transversal de esparcimiento $\sigma$ (no es
    conductividad ni carga superficial, aunque use la misma letra) se
    define como $\sigma=\mathcal P/I$, donde $I$ es la intensidad,
   potencia por unidad de área de la onda incidente.
  \item Verifica que $\sigma$ tiene las unidades correctas.
  \item Como $\mathcal P$ y $I$ son ambos proporcionales a $E^2$, el
    campo se cancela y $\sigma$ depende sólo de las propiedades de la molécula.
  \end{pistas}
  \begin{notas}
  \item La interpretación de $\sigma$ es como el área efectiva de la
    molécula. Como si toda la energía de la onda incidente que cayera sobre el
    área $\sigma$ fuese esparcida, mientras que el resto seguiera su camino.
  \item Nota que $\sigma$ es inversamente proporcional a
    $\lambda^4$. Por tanto, la luz roja es mucho menos esparcida que
    la luz azul.
  \item Cualquier relación de este resultado con el color del cielo y con el de
    los atardeceres, ¡no es coincidencia!
  \end{notas}


\question\label{x} Considera una partícula puntual cargada que se mueve a lo
  largo de una trayectoria $\bm r(t)$. Calcula la potencia instantánea
  que radía.
  \begin{pistas}
  \item Calcula el momento dipolar $\bm p(t)=q\bm r(t)$.
  \item Escríbelo como una superposición de Fourier
    $$\bm p(t)=\int \frac{d\omega}{2\pi}p_\omega e^{-i\omega t}.$$
  \item Para cada $p_\omega$ usa los resultados de los problemas
    anteriores.
  \item El resultado es
    $$\mathcal P(t)=\frac{2}{3}\frac{q^2 a(t)^2}{c^3},$$
    donde $\bm a$ es la aceleración de la partícula
  \item Una forma rápida, no muy formal, de llegar a este resultado es identificar
     $\bm p$ con $q\bm r$, $\omega^2\bm  p$ con $-d^2\bm
    p/dt^2=-q\bm a$ y por tanto $\omega^4 p^2$ es como $q^2 a^2$. El
    factor de 2/3 en lugar de 1/3 viene de que en los problemas
    anteriores calculamos la potencia promedio, y aquí, la potencia
    instantánea.
  \end{pistas}

\end{questions}
\end{document}
