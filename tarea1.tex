\documentclass{exam}
\usepackage{espanhol}
\usepackage{bm}
\usepackage{amsmath}
\usepackage{comment}
%\usepackage{pstricks}
\decimalpoint
\pointname{\ puntos}
\cfoot[]{\thepage}
\begin{document}
\begin{center}
\bf\large Tarea 1\\
Propedéutico Electrodinámica\\
%\date{2020-03-10}
% \today
2020-03-12
\\[20pt]

\end{center}
\hbox to \textwidth{Nombre: \enspace\hrulefill}
Entregar el martes 2020-03-17.

\begin{questions}
  \question Una carga $Q$ se coloca en el centro de un triángulo
  equilatero equidistante de sus tres esquinas en cada una de las
  cuales hay una carga $-e$. La
  fuerza sobre las cargas negativas es 0.
  \begin{parts}
    \part ¿Cuánto vale $Q$?
    \part El sistema ¿está en equilibrio estable?
  \end{parts}
  \question Calcula la energía necesaria para ensamblar un sistema de
  $N$ partículas con cargas $q_n$ y posiciones $\bm r_n$
  ($n=1\ldots N$). Escríbela en términos de las posiciones de cada
  partícula y en términos de los potenciales $\phi_n(\bm r_n)$
  producidos por las partículas $n'\ne n$ en la posición de la
  partícula $n$.
  \question Calcula la energía potencial por ion de un cristal polar 1D
  formado por cargas $\pm q$ alternadas equiespaciadas una distancia $a$
  una de otra.\\Pista: Usa la expansión en potencias de log(1+x).
  \question Usa la ley de Gauss y argumentos de simetría para calcular
  el campo eléctrico $\bm E(\bm r)$ producido por una carga puntual
  $Q$ colocada en el origen.
  \question Considera una línea infinitamente larga en la
  cual se deposita una carga uniformemente con densidad lineal (carga
  por unidad de longitud) $\lambda$.
  Usa la ley de Gauss y argumentos de simetría para calcular el campo
  eléctrico que produce en todo punto.
  \question Considera un cilindro de radio $a$ infinitamente largo en
  cuya superficie se distribuye uniformemente una carga con densidad
  superficial (carga por unidad de área) $\sigma$. Calcula el campo
  eléctrico producido en todo el espacio (afuera y adentro del
  cilindro). Compara tu resultado con el de la pregunta previa.
  \question
  \begin{parts}
    \part Calcula la energía electrostática total de una esfera de
    radio $a$ en cuyo volumen se distribuye uniformememte una carga
    $Q$.
    \part Si un electrón fuera una esfera uniformemente cargada y su
    energía en reposo fuese su energía electrostática, ¿cuál sería su
    radio $r_c$?
    \part Lo mismo pero para una carga $Q$ distribuida únicamente en
    la superficie de la esfera.
  \end{parts}
  \question Considera una esfera hueca de radio interior $a$ y radio
  exterior $b>a$, en cuya pared (entre $a$ y $b$) se distribuye
  uniformemente una carga $Q$. Calcula el potencial eléctrico
  $\phi(\bm r)$ en todo el espacio (en las tres regiones, $r<a$,
  $a<r<b$, $b<r$).
  \question Considera un prisma rectangular de lados $a$, $b$ y $c$
  alineados con los ejes cartesianos $x$, $y$ y $z$ y centrado en el
  origen. Demuestra que en el límite $a,b,c\to 0$,
  $$
  \frac{1}{abc}\int_{\mathrm{prisma}} d\bm s\cdot \bm F(\bm
  r)=\nabla\cdot \bm F(\bm 0)
  $$
  donde $\bm F(\bm r)$ es un campo vectorial arbitrario con
  componentes $F_x$, $F_y$ y $F_z$, la integral
  es sobre las paredes del prisma y $\nabla\cdot\bm F=\partial_x
  F_x+\partial_y F_y+\partial_z F_z$ es la divergencia de $\bm F$ (y
  economicé en la notación).
  \question Demuestra la versión diferencial de la Ley de Gauss,
  $\nabla\cdot \bm E(\bm r)=4\pi\rho(\bm r)$.
\end{questions}
\end{document}
