\documentclass{exam}
\usepackage{espanhol}
\usepackage{bm}
\usepackage{amsmath}
\usepackage{comment}
%\usepackage{pstricks}
\decimalpoint
\pointname{\ puntos}
\cfoot[]{\thepage}
\begin{document}
\begin{center}
\bf\large Tarea 2\\
Propedéutico Electrodinámica\\
\date{2020-03-10}
\today\\[20pt]
\end{center}
\hbox to \textwidth{Nombre: \enspace\hrulefill}
Entregar el martes 2020-03-31.

Como no tenemos clases presenciales, entonces propongo problemas para
desarrollar el temario.

\begin{questions}
  \question Usando la ley de Gauss demuestra que el potencial
  $\phi(\bm r)$ obedece la ecuación de Poisson $\nabla^2\phi(\bm
  r)=-4\pi\rho(\bm r)$

  \question Demuestra el teorema de Gauss: Para {\em todo} (con un
  granito de sal) campo vectorial $\bm F(\bm r)$ y volumen $V$
  con frontera $\partial V$,
  $$
  \int_{\partial V}d\bm a\cdot\bm F(\bm r)=\int_V d^3 r\, \nabla\cdot\bm
  F(\bm r).
  $$
  Notas:
  \begin{itemize}
  \item La superficie $\partial V$ es cerrada y orientable.
  \item $d\bm a$ es el elemento de área, es un vector cuyo tamaño es
    el área de un pedacito de superficie y cuya dirección es el de la
    normal $\hat {\bm n}$ a
    la superficie que sale del volumen. De forma equivalente, $da_x$
    es el área de la proyección del elemento de
    superficie sobre el plano $yz$, $da_y$ corresponde a la proyección
    sobre el plano $zx$ y $da_z$ sobre el plano $xy$. El signo de
    $da_x$ es el de $\hat{\bm  y}\times\hat{\bm z}\cdot\hat{\bm  n}$,
    el de $da_y$ es el de $\hat{\bm  z}\times\hat {\bm x}\cdot\hat{\bm
      n}$,
    y el de $da_z$ es el de $\hat{\bm  x}\times\hat{\bm  y}\cdot\hat{\bm n}$.
  \end{itemize}

  Pistas:
  \begin{itemize}
  \item Lo más fácil, aunque no lo más formal, es dividir el volumen
    V en prismas rectangulares {\em infinitesimales}. Argumenta por qué
    demostrar el teorema para un pequeño prisma implica que se cumple
    para un volumen finito de forma arbitraria.
  \item   Una demostración un poco más convincente es dividir al volumen en
    pequeños tetrahedros y demostrarlo para un tetrahedro, caracterizado
    por un vértice y tres vectores que describan los lados que inciden
    en dicho vértice.
  \item   Otra demostración más consiste en dividir el volumen en regiones
    convexas, proyectarlas sobre el plano $yz$ y cuadricular dicha
    proyección. Cada línea con $y$ y $z$ dadas intersecta a la
    superficie en dos puntos, digamos $x_0$ y $x_1$. Luego, $\Delta a_x
    F_x(x_1,y,z)+\Delta a_x F_x(x_0,y,z)=\Delta y\Delta
    z(F_x(x_1,y,z)-F_x(x_0,y,z))=\Delta y\Delta z\int_{x_0}^{x_1} dx\,
    \partial F_x(x,y,z)/\partial x$ ($\Delta a_x=\pm\Delta y
    \Delta z$ es la componente $x$ de un elemento de área en $(x_1,y,z)$ y
    $(x_0,y,z)$; nota los signos), entonces $\int_{\partial
      V} da_x F_x(\bm r)=\int_V d^3r\, F_x(\bm r)$. Luego, repetir para
    las coordenadas $y$ y $z$.
  \end{itemize}
  \question Demuestra el teorema de Stokes: Para todo campo vectorial
  $\bm F(\bm r)$ y superficie orientable $A$ con frontera $\partial
  A$,
  $$
  \int_{\partial A}d\bm \ell\cdot\bm F(\bm r)=\int_A d\bm
  a\cdot\nabla\times\bm F(\bm r).
  $$
  Notas:
  \begin{itemize}
  \item $\nabla\times \bm F$ es el rotacional con componentes cartesianas
    $(\partial_yF_z-\partial_zF_y$, $\partial_z F_x-\partial_x F_z,
    \partial_x F_y-\partial_y F_x)$ y usé la notación para flojos
    $\partial_i\equiv\partial/\partial r_i$
  \item Por ser orientable, se puede asignar una normal $\hat
    {\bm n}(\bm r)$ en todo punto $\bm r$ de la superficie $A$.
  \item $\partial A$ es una curva cerrada que recorre la frontera de
    la superficie $A$ siguiendo una dirección consistente con la {\em
      regla de la mano derecha}, i.e. si la palma de la mano derecha
    toca dicha frontera y el pulgar apunta hacia la normal, los dedos de
    la mano apuntan en la dirección en que se recorre la curva.
  \item $d\bm \ell$ es el {\em elemento} de longitud a lo largo de la
    curva.
  \item $d\bm a$ es el elemento de área.
  \end{itemize}
  Pistas:
  \begin{itemize}
  \item La forma más sencilla, aunque menos formal y no del todo general, es demostrarlo
    para rectángulos {\em infinitesimales} alineados a los ejes
    cartesianos, por ejemplo, uno de lados $\Delta x$ y $\Delta y$ y
    con elemento de área $\Delta \bm a=\Delta
    x\Delta y\hat{\bm z}$. El problema es que no es fácil generalizar de rectángulos a
    superficies arbitrarias.
  \item Una con algo de talacha pero fácil también es demostrarlo para
    un triángulo {\em infinitesimal} caracterizado por un vértice $\bm
    r_0$ y dos lados pequeños $\bm a$ y $\bm b$. Hay que demostrar que
    la integral de $\bm r_0\to\bm r_0+\bm a\to\bm r_0+\bm b\to\bm r_0$
    es igual a $\nabla\times\bm F(\bm r_0)\cdot\bm a\times\bm b/2$,
    interpretar este resultado y argumentar que toda superficie se
    puede triangular.
  \item En 2D, el teorema de Stokes es idéntico al de
    Gauss, salvo por una rotación. Si describimos una superficie $A$ (o una
    parte de la superficie) como un mapa $\bm r(u,v)$ de una región en el plano
    $u-v$, el teorema de Stokes en 3D se puede deducir del teorema de
    Stokes en 2D haciendo cambios de variable, usando regla de la
    cadena, calculando jacobianos, etc.
  \end{itemize}
  \question Demuestra los teoremas {\em generalizados} de Gauss y de Stokes:
  \begin{parts}
    \part $\int_{\partial V} d\bm a\, f(\bm r)=\int_V d^3r\, \nabla f(\bm
    r)$,
    \part $\int_{\partial V} d\bm a \times \bm F(\bm r)=\int_V d^3r\,
    \nabla \times \bm F(\bm r)$,
    \part $\int_{\partial A}d\bm \ell\cdot\bm F(\bm r)=\int_A (d\bm a
    \times  \nabla)\cdot\bm F(\bm r)$,
    \part $\int_{\partial A}d\bm \ell f(\bm r)=\int_A (d\bm a
    \times  \nabla) f(\bm r)$,
    \part $\int_{\partial A}d\bm \ell\times\bm F(\bm r)=\int_A (d\bm a
    \times  \nabla)\times\bm F(\bm r)$,
  \end{parts}
  donde $V$ es un volumen, $\partial V$ su frontera, $A$ una
  superficie orientable, $\partial A$ su frontera, $\bm F(\bm r)$ un campo
  vectorial, $f(\bm r)$ un campo escalar, $d\bm ell$ un elemento de
  longitud y $d\bm a$ un elemento de superficie.

  Notas:
  \begin{itemize}
  \item Se pueden resumir los resultados anteriores como
    $\int_{\partial V} d\bm a\ldots=\int_Vd^3r\nabla\ldots$ y
    $\int_{\partial A} d\bm \ell\ldots=\int_A(d\bm
    a\times\nabla)\ldots$.
  \end{itemize}

  \question Considera un sistema con carga distribuida {\em
    continuamente} de acuerdo a una densidad $\rho(\bm r)$.
  \begin{parts}
    \part Demuestra que su energía se puede escribir como
    $$
    U=\frac{1}{2}\int_V d^3 r\, \rho(\bm r)\phi(\bm r).
    $$
    \part ¿De dónde viene el factor de 1/2?
    \part Usando la ley de Poisson e {\em integrando por partes}
    demuestra que también podemos escribir
    $$
    U=\frac{1}{8\pi}\int_V d^3r\, |\bm E(\bm r)|^2.
    $$
    \part ¿Qué suposiciones hiciste sobre $\rho$, $\phi$ y/o $\bm E$
    en la superficie $\partial V$. ¿Qué volumen $V$ se debe/puede escoger?
  \end{parts}
  Notas:
  \begin{itemize}
  \item Usaremos con frecuencia integrales por partes en que usamos el
    teorema de Gauss o de Stokes para mover operadores diferenciales
    de un lado a otro. Por ejemplo, $\int_Vd^3r\, f(\bm r)\nabla g(\bm
    r)=\int_Vd^3r\, [\nabla(f(\bm r) g(\bm r))-(\nabla f(\bm r))g(\bm
    r)]=\int_{\partial V} d\bm a\, f(\bm r) g(\bm r)-\int_Vd^3r\,
    (\nabla f(\bm r))g(\bm r)$. En ocasiones podemos eliminar la
    integral de superficie.
  \item Podemos identificar $u(\bm r)=|\bm E(\bm r)|^2/8\pi$ como una
    densidad de energía.
  \end{itemize}
  \question Considera un capacitor de placas paralelas, formado por
  dos placas planas metálicas de lados $a$ y $b$ separadas una
  distancia $d\ll a,b$ en las que se depositan cargas $\pm Q$.
  \begin{parts}
    \part Calcula el campo eléctrico $\bm E(\bm r)$ en todo el
    espacio usando la ley de Gauss.
    \part Calcula la energía electrostática del sistema.
    \part Calcula el cambio de la energía del sistema si una de las
    placas se aleja de la otra una distancia pequeña $\delta d$.
    \part Calcula la fuerza sobre dicha placa comparando el cambio de
    energía con el trabajo realizado.
    \part Calcula la fuerza coulombiana sobre una de las placas.
    \part Discute los últimos dos resultados.
  \end{parts}
  \question Demuestra las identidades
  \begin{parts}
    \part $\nabla (f(\bm r)g(\bm r))=(\nabla f(\bm r)) g(\bm r))+f(\bm r)\nabla g(\bm r),$
    \part $\nabla\cdot(f(\bm r)\bm F(\bm r))=(\nabla f(\bm r))\cdot
    \bm F(\bm
    r)+f(\bm r)\nabla\cdot \bm F(\bm r),$
    \part $\nabla\times(f(\bm r)\bm F(\bm r))=(\nabla f(\bm r))\times
    \bm F(\bm r)+f(\bm r)\times \bm F(\bm r),$
    \part $\nabla\cdot(\bm F(\bm r)\times\bm G(\bm r))=\bm G(\bm
    r)\cdot(\nabla\times \bm F(\bm r))-\bm F(\bm r)\cdot(\nabla\times\bm
    G(\bm r))$
    \part \label{7e}$\nabla\times(\bm F(\bm r)\times \bm G(\bm r))=[\bm F
    \nabla\cdot\bm G(\bm r)-\bm F(\bm r)\cdot\nabla\bm G(\bm r)]-[\bm G(\bm
    r)\nabla\cdot\bm F(\bm r)-\bm G(\bm r)\cdot\nabla\bm F(\bm r)]$,
  \end{parts}
  donde $f(\bm r)$ y $g(\bm r)$ son campos escalares y $\bm F(\bm r)$
  y $\bm G(\bm r)$ son campos vectoriales.

  Pistas:
  \begin{itemize}
  \item Una forma talachuda pero certera es calcular las componentes
    usando notación de índices y la convención de Einstein. Así, por
    ejemplo, el lado derecho del inciso \ref{7e} sería
    $\epsilon_{ijk}\partial_j\epsilon_{klm} F_l G_m$ donde
    $\epsilon_{ijk}$ es el símbolo de Levi-Civita (1 para
    permutaciones pares de $123=xyz$, -1 para impares, 0 en otros
    casos). Entonces, cada derivada se aplica a un simple producto. Se
    pueden entonces usar identidades como
    $\epsilon_{ijk}\epsilon_{klm} = \delta_{ij}^{lm} =
    \delta_{il}\delta_{jm} - \delta_{im}\delta_{jl}$, donde
    $\delta_{ij}^{lm}$ es la delta de cuatro índices y $\delta_{ij}$
    es la delta de Kronecker.
  \item Una forma formal, ingeniosa pero poco ortodoxa, es separar
    $\nabla$  en dos símbolos, $\nabla_F$ que actúe sobre $\bm F$, sin
    importar si quedad a la derecha o a la izquierda, y $\nabla_G$ que
    actúe sólo sobre $\bm G$. Entonces, podemos tratar a $\nabla_F$ y
    $\nabla_G$ como si fueran vectores ordinarios, olvidando que son
    operadores. Luego reordenamos los términos de manera que $\nabla_F$
    quede a la izquierda de $\bm F$ y $\nabla_G$ quede a la izquierda
    de $\bm G$. Finalmente, eliminamos los subíndices $F$ y $G$ por
    ser innecesarios.
  \item Siempre puede uno recurrir a calcular cada componente del lado
    derecho y del lado izquierdo de las ecuaciones anteriores y
    verificar que se cumplen.
  \end{itemize}
\question Utiliza la ley de Gauss y el hecho de que el campo es
conservativo para demostrar que
\begin{parts}
  \part La componente del campo eléctrico tangencial a una superficie
  cualquiera es continua, $\bm E_\|(1)=\bm E_\|(2)$, o $\hat{\bm
    n}\times(\bm E_\|(2)-\bm E_\|(1))=0$, donde los números 2 y 1
  denotan puntos contiguos de uno y otro lado de la superficie y $\hat
  {\bm n}$ es un vector normal a la superficie.
  \part La componente del campo eléctrico perpendicular a una
  superficie son contínuos en ausencia de carga superficial,
  $E_\perp(1)=\bm E_\perp(2)$ o $\hat{\bm n}\cdot(\bm E(2)-\bm E(1))=0$.
  \part En presencia de una carga superficial $\sigma$, hay una discontinuidad
  del campo $4\pi\sigma$, i.e. $\hat{\bm n}\cdot(\bm E(2)-\bm E(1))=4\pi\sigma$
  donde el vector normal va del lado 1 al lado 2 de la superficie.
  \part El potencial es continuo $\phi(2)=\phi(1)$ (en ausencia de una
  distribución singular de dipolos).
\end{parts}
Pista:
\begin{itemize}
\item Usa pequeños cilindros gaussianos con tapas paralelas a la superficie,
una de cada lado, y pequeños rectángulos con dos lados normales a la
superficie y dos a lo largo de la superficie, uno a cada lado. Usa los
teoremas de Stokes y de Gauss.
\item Otra alternative consiste en, sin pérdida de generalidad,
  colocar el plano $xy$ tangente al punto de la superficie que nos
  interesa e integrar $\nabla\cdot\bm E$ y $\nabla\times\bm E$ de
  $z=0-\epsilon$ hasta $z=0+\epsilon$, y tomar el límite
  $\epsilon\to0$.
\end{itemize}
\question
\begin{parts}
  \part Argumenta que en el interior de un conductor el campo
  electrostático es nulo.
  \part Demuestra que, por tanto, la densidad de carga es nula en el
  interior del conductor.
  \part Justo afuera del conductor el campo es
  perpendicular a la superficie, $\bm E_\|=0$.
  \part El conductor es equipotencial.
  \part Muestra que el campo perpendicular justo afuera es $E_\perp=4\pi\sigma$, con $\sigma$
  la densidad de carga superficial.
\end{parts}
\question Demuestra que dada una densidad de carga $\rho(\bm r)$
dentro de un volumen $V$ y el potencial $\phi(\bm r)$ en la frontera
$\partial V$ de dicho volumen, el potencial es único en todo el
volumen $V$.

Pistas:
\begin{itemize}
\item El potencial obedece la ecuación de Poisson.
\item Suponiendo que hubiera dos soluciones $\phi_1(\bm r)$ y
  $\phi_2(\bm r)$ de la ec. de Poisson, su diferencia $\Phi(\bm
  r)\equiv\phi_2(\bm r)-\phi_1(\bm r)$ obedecería la ecuación de
  Laplace y valdría cero en la superficie $\partial V$.
\item Define $\bm F=-\nabla\Phi$.
\item Demuestra que $\int_Vd^3r\, F^2=0$ integrando por
  partes y usando las condiciones de frontera y la ecuación de
  Laplace.
\item Concluye que $\bm F=0$, por tanto $\Phi$ es constante.
\item Demuestra que la constante es cero.
\end{itemize}
Nota: La misma demostración se puede aplicar al caso en que conocemos
$E_\perp$, concluyendo en dicho caso que $\phi$ es única hasta una
constante aditiva, pero $\bm E$ sería único.
\question Se coloca una carga $q$ sobre el eje $z$ en el vacío a una distancia $d$
arriba de una superficie metálica plana {\em aterrizada} (i.e., con potencial
$\phi(r)=0$) que descanza en el plano $xy$.
\begin{parts}
  \part Demuestra que el potencial en la región $z>0$ es el mismo que
  el potencial coulombiano que produce la carga $q$ más el que produce
  una carga ficticia $-q$ colocada en el eje $z$ a una distancia $d$
  abajo de la superficie.
  \part ¿Cuánto vale el potencial en la región $z<0$?
  \part ¿Cuánto vale la carga $\sigma(x,y)$ superficial inducida en la
  superficie del metal?
\end{parts}
\question Se coloca una carga $q$ en el eje $z$ a una distancia $d$
arriba del origen y se coloca una esfera metálica aterrizada de radio $a<d$
centrada en el origen.
\begin{parts}
  \part Demuestra que el potencial en la región $r>a$ es el mismo que
  el potencial coulombiano que produce la carga $q$ más el que produce
  una carga ficticia $q'=-q a/d$ colocada en el eje $z$ a una distancia $d'=a^2/d$
  arriba del origen.
  \part ¿Cuánto vale el potencial en la región $r<a$?
\end{questions}


\end{document}
