\documentclass{exam}
\usepackage{espanhol}
\usepackage{bm}
\usepackage{amsmath}
\usepackage{comment}
%\usepackage{pstricks}
\decimalpoint
\pointname{\ puntos}
\cfoot[]{\thepage}
\begin{document}
\begin{center}
\bf\large Tarea 4\\
Propedéutico Electrodinámica\\
%\date{2020-04-12}
%\today
2020-04-12\\[20pt]
\end{center}
\hbox to \textwidth{Nombre: \enspace\hrulefill}
Entregar el viernes 2020-04-17.

\begin{questions}
\question Considera un sistema formado por $n$ partículas por unidad
  de volumen, cada una con carga $q$ y todas moviéndose al unísono con
  velocidad $\bm v$. Considera una pequeña superficie con área $\Delta
  a$ y con un vector normal unitario $\hat {\bm n}$. Demuestra que en
  un tiempo $\Delta t$ la cantidad de carga $\Delta Q$ que atraviesa
  la superficie en la dirección de $\hat{\bm n}$  es
  $$
  \Delta Q=(n q \bm v\Delta t)\cdot(\Delta a \hat{\bm n})
  $$

  Pistas:
  \begin{itemize}
  \item Construye un cilindro oblicuo cuya {\em tapa} sea la pequeña
    superficie dada y cuyos lados esten dados por el vector $\bm v\Delta
    t$, de manera que $\bm v\cdot\hat{\bm n} \ge 0$.
  \item El volumen de dicho cilindro es base por altura, $(\bm v\Delta
    t)\cdot(\Delta a\hat{\bm n})$.
  \item Calcula cuánta carga contiene dicho cilindro.
  \item  Muestra que toda la carga que contenida en dicho cilindro al
    tiempo $t$ habría atravesado la tapa al tiempo $t+\Delta t$,
    formando un nuevo cilindro cuya {\em base} es la pequeña
    superficie.
  \end{itemize}

  Notas:
  \begin{itemize}
  \item El problema así planteado es irrealista. No importa. Abajo lo arreglaremos.
  \item Podemos definir in {\em vector} área $\Delta \bm a=\Delta
    a\hat{\bm n}$.
  \item Podemos definir una {\em densidad de corriente} $\bm j=n q\bm
    v$.
  \item Podemos escribir el resultado arriba como $\Delta Q=\bm
    j\cdot\hat{\Delta \bm a}\Delta t$.
  \end{itemize}

  \question Considera un sistema formado por $n_\alpha$ partículas cargadas por
    unidad de volumen, cada una con una carga $q_\alpha$ y moviéndose
    con velocidad $\bm v_\alpha$, donde
    $\alpha=1,2\ldots$ denota el tipo de partícula (por ejemplo,
    electrones y diversos tipos de iones).
    \begin{parts}
    \part Muestra que la densidad de carga es $\rho=\sum_\alpha
      n_\alpha q_\alpha$.
    \part Muestra que la densidad de corriente es $\bm j=\sum_\alpha
      n_\alpha q \bm v_\alpha$.
    \end{parts}

    Pistas:
    \begin{itemize}
    \item Muestra que un pequeño volumen $\Delta V$ contiene una carga
      $\Delta Q_{\Delta V}=\rho\Delta V$.
    \item Muestra que una carga $\Delta Q_{\Delta \bm a,\Delta t}=\bm
      j\cdot\Delta\bm a\Delta t$ atraviesa una pequeña superficie
      $\Delta \bm a$ en la dirección de su normal en un tiempo $\Delta t$.
    \item Usa el resultado del problema anterior para cada tipo de cargas.
    \end{itemize}

    Notas:
    \begin{itemize}
    \item Cuando hay más de un tipo de cargas no es cierto que $\bm
      j=\rho \bm v$; cada tipo de cargas contribuye de manera distinta
      a la densidad de carga y a la densidad de corriente.
    \item Puede haber una corriente finita en un medio neutral.
    \item Típicamente, los metales son materiales que pueden conducir
      corriente eléctrica $\bm j\ne \bm 0$ a pesar de ser neutros
      $\rho=0$, i.e., tienen el mismo número de electrones de
      conducción móviles que de protones inmovilizados en los núcleos.
    \end{itemize}
  \question Considera un volumen $V$ cuyas cargas se mueven dando
    origen a una densidad local de corriente $\bm j(\bm
    r,t)$. Demuestra que la carga que sale de dicho volumen durante un
    pequeño tiempo $\Delta$ es
    $$
    \Delta Q_{\mathrm{sale}}=\int_{\partial V}d\bm a \bm j(\bm r, t) \Delta t,
    $$
    donde $\partial V$ denota la frontera del volumen, donde $d\bm
    a=da\hat{\bm n}$, $da$ es un elemento de área y $\hat{\bm n}$ es
    la normal a la superficie que apunta hacia afuera.

    Pista:
    \begin{itemize}
    \item Calcula la carga que atraviesa cada elemento de área
      empleando los resultados de los problemas anteriores.
    \end{itemize}
    Notas:
    \begin{itemize}
    \item La densidad de corriente se relaciona a la densidad y
      movimiento de los portadores de carga como en el problema
      anterior.
    \item La densidad de número de los portadores y su velocidad pueden
      depender de la posición y del tiempo, por lo cual $\bm j$
      depende en general de $\bm r$ y de $t$.
    \item La frontera $\partial V$ de un volumen $V$ es una superficie
      cerrada y {\em orientable}.
    \end{itemize}
  \question Muestra que {\em la ley de conservación de la carga} se
    puede expresar como
    $$
    \frac{d}{dt}Q_V(t)+\int_{\partial V}d\bm a\cdot\bm j(\bm r,t)=0,
    $$
    Notas:
    \begin{itemize}
    \item La carga no se crea ni se destruye. La única manera como
      puede salirse carga del interior de un volumen es atravesando su
      superficie.
    \item Se pueden crear y destruir partículas cargadas, pero siempre
      por pares, acompañadas de la creación o destrucción en el mismo
      punto y al mismo tiempo de una partícula con la carga
      opuesta. Se crean y destruyen pares de partículas de carga
      opuesta.
    \end{itemize}
  \question Muestra que la ley de conservación de la carga también se puede
    escribir como
    $$ \frac{\partial}{\partial t}\rho(\bm r,t)+\nabla\cdot\bm j(\bm
    r,t)=0.$$
    Pistas:
    \begin{itemize}
    \item En el problema anterior escribe $Q_V(t)$ en términos de
      $\rho(\bm r,t)$.
    \item Usa el teorema de Gauss.
    \item Demuestra que $\int_Vd^3r(\partial\rho+\nabla\cdot\bm j)=0$
      {\em para todo volumen} $V$, y por tanto el integrando es nulo.
    \end{itemize}
    Nota:
    \begin{itemize}
    \item Éstas últimas ecuaciones se conocen como {\em ecuación de
        continuidad}.
    \end{itemize}

  \question Demuestra que en un sistema estacionario
    \begin{parts}
    \part $\int_{\partial V}d\bm a\cdot\bm j(\bm r)=0$ donde
      $\partial V$ es la frontera de un volumen arbitrario $V$ fijo.
    \part $\nabla\cdot\bm j(\bm r)=0$.
    \end{parts}
    Notas:
    \begin{itemize}
    \item Un sistema se llama {\em estático} cuando nada se mueve.
    \item Un sistema se llama {\em estacionario} cuando ninguna
      cantidad (macroscópica) depende del tiempo.
    \end{itemize}

  \question Considera el siguiente modelo simplificado de metal,
    formado por $n$ iones positivos y $n$ electrones por unidad de
    volumen. Cada ion tiene carga $e$ y cada electrón carga $-e$. Cada
    electrón se mueve libremente durante cierto tiempo azaroso $\Delta
    t$ hasta que choca. Inmediatamente después del choque los
    electrones emergen moviendose en una dirección totalmente al
    azar.
    \begin{parts}
    \part Demuestra que al aplicar un campo eléctrico $\bm E$ se
      induce una densidad de corriente $\bm j=\sigma \bm E$, donde
      $\sigma$ es la {\em conductividad} dada por
      $$
      \sigma=\frac{ne^2\tau}{m},
      $$
    \end{parts}
    donde $\tau$ es el tiempo promedio transcurrido desde el último
    choque hasta el momento de observación.

    Pistas:
    \begin{itemize}
    \item Integra la ecuación de movimiento de un electrón desde que
      emerge de su choque previo hasta el momento de observación.
    \item Promedia el resultado sobre todas las posibles velocidades
      con las que pudo haber emergido y sobre todos los momentos en
      los que pudo haber sido el choque.
    \item Argumenta por qué el promedio de la velocidad inicial
      (pasando el choque anterior) es cero, aunque el promedio de la
      rapidez no.
    \item Sustituye la velocidad promedio en las expresiones
      anteriores para la densidad de corriente.
    \end{itemize}
    Notas:
    \begin{itemize}
    \item El modelo es malón, pero el resultado es razonablemente
      bueno y útil.
    \item Cada material tiene su propia conductividad. Es de las propiedades
      que más variaciones muestra entre distintos materiales. Por
      ejemplo, va desde $5\times 10^{17}s^{-1}$ para la plata hasta
      $10^{-15}s^{-1}$ para el teflón.
    \item  Demuestra que las unidades de la conductividad en el
      sistema CGS son $s^{-1}$.
    \end{itemize}
  \question Considera un alambre cilíndrico conductor en el cual circula una corriente
    eléctrica con densidad $\bm j(\bm r,t)$. Considera dos superficies
    $\mathcal S_1$ y $\mathcal S_2$ que cortan el alambre con una
    normal que apunta a lo largo de la dirección que lleva de
    $1$ a $2$ a lo largo del alambre. Define
    $I_1(t)=\int_{\mathcal S_1}d\bm a\cdot \bm j(\bm r,t)$ e
    $I_2(t)=\int_{\mathcal S_s}d\bm a\cdot\bm j(\bm r,t)$ como las
    corrientes totales que atraviesan las superficies 1 y 2. Demuestra
    que en el caso estacionario $I_1=I_2$.

    Pistas:
    \begin{itemize}
    \item Considera una superficie $\mathcal S$ en forma de cilindro,
      quizás chueco, cuya base sea
      $\mathcal S_1$, cuya tapa sea $\mathcal S_2$ y cuyas paredes
      estén pegadas a las paredes del alambre pero por fuera. Dicha
      superficie encierra totalmente a un fragmento de alambre.
    \item Demuestra que $\int_{\mathcal S}d\bm a\cdot\bm
      j=\int_{\mathcal S_2}d\bm a\cdot\bm j - \int_{\mathcal
        S_1}d\bm a\cdot\bm j=I_2-I_1$.
    \item Usa la ley de conservación de la carga en el caso estacionario.
    \end{itemize}
    Notas:
    \begin{itemize}
    \item Como $I_1=I_2$ sin importar en qué parte del alambre hacemos
      los cortes 2 y 1, en el caso estacionario podemos hablar de una
      corriente $I$ que es la misma a todo lo largo del alambre.
    \item El resultado anterior es válido aunque el alambre cambie de
      sección transversal a su largo.
    \item Por lo tanto, una corriente estacionaria sólo puede circular
      en un alambre finito que se cierra sobre sí mismo o en un
      alambre infinitamente largo.
    \end{itemize}


  \question Usando la ley de Biot y Savart, calcula la fuerza entre
    dos alambres conductores $\mathcal A_1$ y $\mathcal A_2$ delgados
    rectos paralelos de longitud $L$ y
    separados una distancia $d\ll L$ en los que circulan corrientes
    estacionarias $I_1$ y $I_2$.

    Pistas:
    \begin{itemize}
    \item Coloca el alambre 1 a lo largo del eje $z$ centrado en el origen.
    \item Coloca el alambre 2 paralelo al eje $z$ y centrado en el
      punto $(d,0,0)$.
    \item Calcula el campo magnético producido por el alambre 1
      evaluado a lo largo del alambre 2 usando
      $$\bm B_1(\bm
      r)=\frac{I_1}{c}\int_{\mathcal A_1} d\bm l_1\times \frac{\bm r-\bm r_1}{|\bm
        r-\bm r_1|^3}.$$
    \item Para simplificar el cálculo anterior {\em adivina} en qué
      dirección apuntará el resultado y sólo calcula la componente
      relevante de $\bm B_1$. Ignora efectos de borde y aproxima el
      resultado usando $d\ll L$.
    \item Nota de qué variables depende realmente $\bm B_1(\bm r)$.
    \item Calcula la fuerza usando
      $$\bm F_{21}=\frac{I_2}{c}\int_{\mathcal A_2} d\bm l_2\times \bm B(\bm r_2).$$
    \item Simplifica este último cálculo.
    \end{itemize}
    Notas:
    \begin{itemize}
    \item Nota que el problema está mal planteado. Los conductores no
      pueden ser de longitud $L$ finita y llevar una corriente
      estacionaria.
    \item En la ley de Biot y Savart aparecen integrales sobre
      trayectorias cerradas,
      $$
      \bm F_{21}=\frac{I_2I_1}{c^2}\oint_{\mathcal
        C_2}d\bm l_2\times\oint_{\mathcal C_1}d\bm l_1\times \frac{\bm r_2-\bm r_1}{|\bm
        r_2-\bm r_1|^3}.
      $$
      En este caso se pueden cambiar las integrales por integrales
      sobre circuitos abiertos, i.e., las partes rectas de los
      circuitos enfrentadas. ¿Por qué?
    \item Nota el parecido entre esta ley de Biot y Savart y la ley de
      Coulomb
      $$
      \bm F_{21}=\int_1 dq_2 \int_2 dq_1 \frac{\bm r_2-\bm r_1}{|\bm r_2-\bm r_1|^3}.
      $$
      En el caso de líneas de carga reemplazaríamos
      $dq_1\to\lambda_1dl_1$ y $dq_2\to\lambda_2dl_2$.
      ¿Cuáles son las diferencias?
    \item En analogía al caso electrostático, la fuerza resulta
      proporcional a la longitud $L$ y a las corrientes $I_1$ e $I_2$,
      e inversamente proporcional a la distancia $d$.
    \item Si las corrientes son paralelas, la fuerza ¿es atractiva o
      repulsiva?. Y ¿si las corrientes son antiparalelas?
    \end{itemize}
  \question Encuentra las unidades de $c$ a partir de la expresión para
    $\bm F_{21}$ arriba.
  \question Encuentra
    \begin{parts}
    \part las unidades de $\bm B$ a partir de la ley de Biot-Savart y
    \part comparalas con las unidades del campo eléctrico $\bm E$.
    \end{parts}
  \question Demuestra que la ley de Biot y Savart es compatible con la
    segunda ley de Newton.

    Pistas:
    \begin{itemize}
    \item Intercambia los índices $1\leftrightarrow2$ en la ley
      de Biot y Savart
      $$
      \bm F_{21}=\frac{I_2I_1}{c^2}\oint_{\mathcal
        C_2}d\bm l_2\times\oint_{\mathcal C_1}d\bm l_1\times \frac{\bm r_2-\bm r_1}{|\bm
        r_2-\bm r_1|^3}.
      $$
      para obtener $\bm F_{12}$.
    \item Suma $\bm F_{21}+\bm F_{12}$ y usa identidades del álgebra y
      del cálculo vectorial para demostrar que el resultado es cero.
    \end{itemize}

  \question Considera una espira circular de radio $a$ que descansa en
    el plano $xy$ y centrada en el origen, formada por un alambre
    conductor delgado en el que circula una corriente $I$. Utilizando
    la ley de Biot y Savart calcula el campo magnético a lo largo del
    eje $z$.

    Pista:
    \begin{itemize}
    \item Identifica la única componente distinta de cero para
      simplificar tu cálculo.
    \end{itemize}
    Notas:
    \begin{itemize}
    \item Nota que el campo es máximo en el origen.
    \item Nota que a distancias grandes el campo decae como $1/z^3$,
      similar al caso de un dipolo eléctrico.
    \end{itemize}

  \question Considera dos espiras de radio $a$ como las del problema
    anterior, ambas llevando una corriente $I$ y
    separadas una distancia $d$, descansando en los planos $z=\pm d/2$. Encuentra cuánto
    debe valer $d$ para que el campo en el origen sea lo más constante
    posible.

    Pista:
    \begin{itemize}
    \item Calcula la segunda derivada $\partial^2/\partial z^2
      B_z(0,0,z)|_{z=0}$ y encuentra para que valor de $d$ ésta vale
      cero.
    \end{itemize}
    Notas:
    \begin{itemize}
    \item Nota que si las espiras están muy separadas ($d$ grande) el campo tendrá
      dos máximos, uno en cada centro, y un mínimo entre ellas.
    \item En cambio, si están casi pegadas, el campo tendrá un máximo
      en su centro común.
    \item En el primer caso la segunda derivada será positiva en el
      origen.
    \item En el segundo caso la segunda derivada será negativa en el
      origen.
    \item Entonces, para algún valor intermedio, la segunda derivada deberá ser
      cero.
    \item Argumenta por qué la primera y la tercera derivada son cero.
    \item Por lo tanto, haciendo una expansión de Taylor, el campo es de la forma
      $B_z(z)=B_z(0)+\mathrm{constante}\times z^4$.
    \item Estas espiras se conocen como {\em bobinas de Helmholtz}.
    \end{itemize}
  \question A partir de la ley de Biot y Savart demuestra que la
    fuerza $\bm f(\bm r)$ por unidad de volumen en una región en que
    circula una corriente descrita por la densidad $\bm j(\bm r)$ es
    $$
    \bm f(\bm r)=\frac{1}{c}\bm j(\bm r)\times \bm B(\bm r).
    $$

    Pistas:
    \begin{itemize}
    \item Considera un fragmento muy pequeño de
      longitud $d\bm l$ y de sección transversal $S$ tomado de un circuito $\mathcal
      C$ conductor delgado. La fuerza total
      sobre el circuito, de acuerdo a Biot-Savart,
      $$
      \bm F=\frac{I}{c}\oint_{\mathcal C} d\bm l\times\bm B(\bm r),
      $$
      es como una suma de términos de la forma
      $d\bm F=(I/c)d\bm l\times \bm B(\bm r)$.
    \item Pero la corriente $I$ y
      la densidad de corriente están relacionadas $I=jS$. Luego
      $d\bm F=(1/c)S j d\bm l\times \bm B(\bm r)=(1/c)S dl \bm j
      \times \bm B(\bm r)$.
    \item Nota el sutil truco $jd\bm l=dl\bm j$, pues $\bm j$ y $d\bm
      l$ son paralelos.
    \item Identificando $dlS$ con un elemento de volumen del conductor
      llegamos al resultado.
    \end{itemize}
  \question Demuestra que la fuerza sobre una carga $q$ situada en
    $\bm r$ que se mueve con velocidad $\bm v$ es
    $$
    \bm F=q\frac{\bm v}{c}\times\bm B(\bm r).
    $$
    Pistas:
    \begin{itemize}
    \item En un problema previo obtuvimos la fuerza por unidad de
      volumen $f=(1/c)\bm j\times B$,
    \item y en otro problema anterior vimos que $\bm j=nq\bm v$ para
      un sistema de $n$ partículas por unidad de volumen cada una con
      carga $q$.
    \item Dividiendo $f$ entre $n$ obtenemos la fuerza sobre una partícula.
    \end{itemize}
    Notas:
    \begin{itemize}
    \item Esta expresión se conoce como la {\em Fuerza de Lorentz}.
    \item Si además hay campo eléctrico, la fuerza total sería $\bm
      F=q(\bm E(\bm r)+\frac{\bm v}{c}\times\bm B(\bm r))$.
    \end{itemize}

  \question Una partícula se mueve al tiempo $t=0$ con velocidad $\bm v$ en presencia
    de un campo magnético $\bm B=B\hat{\bm z}$.
    \begin{parts}
    \part Describe su movimiento subsecuente.
    \part Calcula la velocidad angular $\omega_c$ de su giro.
    \part Calcula el radio $r_c$ de su giro.
    \end{parts}
    Pistas:
    \begin{itemize}
    \item Escribe la ecuación de movimiento.
    \item Separa la ecuación a lo largo de $z$ y las ecuaciones en el
      plano $xy$.
    \item Muestra que $v_z$ es constante.
    \item Desacopla las ecuaciones en el plano $xy$ sustituyendo una
      en la derivada de la otra.
    \item Resuelve la ecuación resultante para $v_x$.
    \item Sustituye para obtener $v_y$.
    \item Integra el resultado en el tiempo para obtener la
      trayectoria $\bm r(t)$.
    \item Identifica el movimiento como una hélice que avanza en $z$
      mientras gira en $xy$.
    \item Identifica la velocidad angular $\omega_c$.
    \item Identifica el radio de giro $r_c$.
    \end{itemize}
    Nota:
    \begin{itemize}
    \item $\omega_c$ se conoce como {\em frecuencia de ciclotrón}.
    \end{itemize}
  \question Considera una partícula quieta en el origen al tiempo
    $t=0$, en presencia de un campo eléctrico $\bm E=E\hat{\bm x}$ y
    un campo magnético $\bm B=B\hat{\bm y}$ uniformes y constantes.
    \begin{parts}
    \part Obtén y describe el movimiento resultante $\bm r(t)$.
    \part Muestra que la partícula se mueve en promedio con velocidad
      constante $\bm v_d=(E/B)c\hat{\bm z}$.
    \end{parts}
    Notas:
    \begin{itemize}
    \item El movimiento es una cicloide, un desplazamiento a velocidad
      constante sumado a una rotación.
    \item El resultado anterior sólo vale cuando $B\gg E$, pero esa
      limitación tiene que ver con la teoría de la relatividad.
    \end{itemize}
  \question Una corriente estacionaria se distribuye de acuerdo a una
    densidad $\bm j(\bm r)$. A partir de la ley de Biot-Savart
    demuestra que
    $$
    \bm B(\bm r)=\frac{1}{c}\int d^3r'\,\bm j(\bm r')\times\frac{\bm
      r-\bm r'}{|\bm r-\bm r'|^3}.
    $$
    Pistas:
    \begin{itemize}
    \item Para el caso del campo producido por un circuito $\mathcal C$, interpreta
      $$
      \bm B(\bm r)=\frac{I}{c}\int_{\mathcal C}d\bm l'\times \frac{\bm
        r-\bm r'}{|\bm r-\bm r'|^3}
      $$
      como una suma de términos de la forma
      $$
      d \bm B(\bm r)=\frac{I}{c} d\bm l'\times \frac{\bm
        r-\bm r'}{|\bm r-\bm r'|^3}.
      $$
    \item Si el alambre en $\bm r'$tiene una sección transversal $S'$, escribe
      $I=j(\bm r')S$.
    \item Sustituye
      $$
      d \bm B(\bm r)=\frac{1}{c} S' j(\bm r') d\bm l'\times \frac{\bm
        r-\bm r'}{|\bm r-\bm r'|^3}.
      $$
    \item Intercambia el caracter vectorial $j(\bm r') d\bm l'=dl'\bm j(\bm r')$
    \item e identifica $S' dl'=dV'$ en términos de un elemento de
      volumen del alambre,
      $$
      d \bm B(\bm r)=\frac{1}{c} d^3r' \bm j(\bm r') \times \frac{\bm
        r-\bm r'}{|\bm r-\bm r'|^3}.
      $$
    \item Finalmente, integra sobre el volumen del alambre. Para
      corrientes distribuidas sobre un volumen extendido, interpreta
      este como la unión de muchos {\em tubos de flujo}, aplica el
      resultado anterior para cada uno y suma para obtener una
      integral sobre todo el espacio.
    \end{itemize}
  \question Demuestra que podemos escribir
    $$ \bm B(\bm r)=\nabla\times \bm A(\bm r)$$
    donde
    $$ \bm A(\bm r)=\frac{1}{c}\int d^3r' \frac{\bm j(\bm r')}{|\bm
      r-\bm r'|}.
    $$
    Pistas:
    \begin{itemize}
    \item Reconoce a $(\bm r-\bm r')/|\bm r-\bm r'|^3$ en el problema
      anterior como análogo a un campo Coulombiano.
    \item Dicho campo se deriva de un potencial Coulombiano $1/|\bm
      r-\bm r'|$, i.e.,
      $(\bm r-\bm r')/|\bm r-\bm r'|^3=-\nabla (1/|\bm r-\bm r'|)$.
    \item Saca $\nabla$ de la integral del problema anterior, donde
      $\bm j(\bm r')$ no depende de $\bm r$
    \item Puedes usar la identidad $\nabla \times f(\bm r) \bm
      F=(\nabla f(\bm r))\times \bm F$ donde $f$ es un campo escalar y
      $\bm F$ un vector constante.
    \end{itemize}
  \question Demuestra la ley de Gauss magnética $\nabla\cdot \bm B(\bm r)=0$.

    Pista:
    \begin{itemize}
    \item Usa el resultado del problema anterior.
    \end{itemize}

  \question Demuestra que para toda superficie cerrada orientable
    $\partial V$
    $$\int_{\partial V}d\bm a\cdot\bm B(\bm r)=0.$$

    Pistas:
    \begin{itemize}
    \item Usa el resultado del problema anterior
    \item y el teorema de Gauss.
    \end{itemize}
  \question Demuestra que $\nabla\cdot\bm A(\bm r)=0$.

    Pistas:
    \begin{itemize}
    \item Toma la divergencia de la expresión anterior
      $$ \nabla\cdot\bm A(\bm r)=\frac{1}{c}\int d^3r' \bm j(\bm r')\cdot\nabla\frac{1}{|\bm
        r-\bm r'|}.
      $$
    \item Cambia la derivada sobre $\bm r$ por una derivada sobre $\bm
      r'$ aprovechando que $1/|\bm r-\bm r'|$ sólo depende de la
      diferencia $\bm r-\bm r'$.
    \item Integra
      $$ \nabla\cdot\bm A(\bm r)=-\frac{1}{c}\int d^3r' \bm j(\bm
        r')\cdot\nabla'\frac{1}{|\bm r-\bm r'|}.
      $$
      por partes.
    \item Usa el teorema de Gauss para escribir una parte
      $$-\frac{1}{c}\int_V d^3r' \nabla'\cdot \frac{\bm j(\bm
        r')}{|\bm r-\bm r'|}.
      $$
      como una integral de superficie. ¿Sobre qué volumen $V$ y sobre
      qué superficie $\partial V$ se realiza dicha integral?
    \item Argumenta por qué dicha interal es nula.
    \item Para la otra parte usa la ecuación de continuidad.
    \end{itemize}
  \question Demuestra que
    $$\nabla^2\bm A(\bm r)=-\frac{4\pi}{c}\bm j(\bm r).$$

    Pista:
    \begin{itemize}
    \item Usa la analogía con los resultados de electrostática
      $\phi(\bm r)=\int d^3r'\,\rho(\bm r')/|\bm r-\bm r'|$ y
      $\nabla\phi(\bm r)=-4\pi\rho(\bm r)$ y las ecuaciones que cumple
      cada componente de $\bm A(\bm r)$.
    \end{itemize}
  \question Demuestra la {\em ley de Ampère} diferencial
    $$\nabla\times\bm B(\bm r)=\frac{4\pi}{c}\bm j(\bm r)$$
    Pistas:
    \begin{itemize}
    \item Escribe $\bm B$ en términos de $\bm A$ y toma el rotacional.
    \item Usa la identidad $\nabla\times(\nabla\times \bm
      A)=\nabla\nabla\cdot\bm A-\nabla^2\bm A$.
    \item Sustituye los resultados de los problemas anteriores.
    \end{itemize}
  \question Demuestra la forma integral de la ley de Ampère
    $$
    \int_{\partial \mathcal A} d\bm l\cdot\bm B(\bm
    r)=\frac{4\pi}{c}I_{\mathcal A},
    $$
    donde $\mathcal A$ es una superficie orientable, $\partial
    \mathcal A$ es su frontera, una curva que se recorre en la
    dirección que dicta la ley de la mano derecha, y
    $$
    I_{\mathcal A}=\int_{\mathcal A}d\bm a\cdot\bm j(\bm r)
    $$
    es la corriente que atraviesa la superficie $\mathcal A$.

    Pista:
    \begin{itemize}
    \item Integra $\nabla\times\bm B$ sobre la superficie $\mathcal A$
      y usa el teorema de Stokes.
    \end{itemize}
  \question Considera un alambre recto delgado infinitamente largo que lleva
    una corriente $I$. Encuentra el campo $\bm B(\bm r)$ que produce a su
    alrededor.

    Pistas:
    \begin{itemize}
    \item Usa argumentos de simetría para concluir que el campo rodea
      al alambre de acuerdo a la ley de la mano derecha y su magnitud
      no depende del ángulo.
    \item Usa la ley de Ampère integral en un disco circular de radio
      $r$ centrado en el alambre y en un plano perpendicular al mismo.
    \item Concluye que $2\pi r B_\theta(a)=4\pi I/c$ y despeja.
    \item Concluye que $B_\theta(r)=2 I/rc$.
    \end{itemize}
  \question Considera un alambre recto cilíndrico de radio $a$
    infinitamente largo que lleva una corriente $I$ uniformemente
    distribuida en su sección transversal. Encuentra el campo $\bm
    B(\bm r)$ en todo el espacio.

    Pistas:
    \begin{itemize}
    \item Sigue los pasos del problema anterior.
    \item Si empleas un disco de radio $r>a$  nada cambia.
    \item Si empleas un disco de radio $r<a$, la corriente que lo
      atraviesa es una fracción $r^2/a^2$ de la corriente $I$.
    \end{itemize}

  \question Considera una bobina circular recta formada por un alambre
    que se enrrolla uniformemente formando $N\ll1$
    espiras alrededor de un cilindro de radio $a$ y altura $\ell\gg
    a$. Despreciando efectos de borde, demuestra que el campo en su
    interior es constante, apunta a lo largo del eje y su tamaño es
    $B=4\pi N I/\ell c$. Su dirección está dada por la ley de la mano
    derecha con el pulgar hacia el campo y los demas dedos girando con
    la corriente.

    Pistas:
    \begin{itemize}
    \item Argumenta por simetría, usando la ley de Gauss magnética y
      la ley de
      Ampère sobre discos coaxiales que el campo debe ser axial.
    \item Integra la ley de Ampère sobre pequeños rectángulos con dos
      lados paralelos al eje y dos lados radiales.
    \item Concluye que el campo axial en el exterior es constante y
      por condiciones de contorno, nulo.
    \item Concluye que el campo axial en el interior es constante.
    \item Integrando sobre un rectángulo que tenga un lado dentro y un
      lado fuera del cilindro, obtén el valor del campo dentro.
    \end{itemize}



  \end{questions}
\end{document}
