\documentclass{exam}
\usepackage{espanhol}
\usepackage{bm}
\usepackage{amsmath}
\usepackage{comment}
%\usepackage{pstricks}
\decimalpoint
\pointname{\ puntos}
\cfoot[]{\thepage}
\begin{document}
\begin{center}
  \bf\large Tarea 5\\
  Propedéutico Electrodinámica\\
  2020-04-19\\[20pt]
\end{center}
\hbox to \textwidth{Nombre: \enspace\hrulefill}
Entregar el viernes 2020-04-24.

\begin{questions}
\question Demuestra que
  \begin{parts}
    \part $$\int d^3 r\, \bm j(\bm r)=0,$$ y
    \part $$\int d^3 r\, (j_i(\bm r)r_j+r_ij_j(\bm r))=0,$$
  \end{parts}
  para una distribución de corriente estacionaria $\bm j(\bm r)$
  localizada en una región finita del espacio.

  Pistas:
  \begin{itemize}
  \item Escribe $j_i=j_k\partial_k r_i$ y $j_j=j_k\partial_k r_j$, con
    $\partial_k\equiv\partial/\partial r_k$.
  \item Manipula los integrandos arriba para escribirlos en términos
    de $j_i=\nabla\cdot(\bm j r_i)-(\nabla\cdot\bm j)r_i$, y
    $j_i r_j+r_i j_j=\nabla\cdot(\bm j r_ir_j)-(\nabla\cdot\bm
    j)r_ir_j$.
  \item Integra los términos $\nabla\cdot(\ldots)$ usando el teorema de
    Gauss.
  \item Integra los términos $(\nabla\cdot\bm j)\ldots$ empleando la
    ecuación de continuidad para sistemas estacionarios.
  \end{itemize}

\question Considera una distribución de corriente estacionaria $\bm
  j(\bm r)$ localizada en el interior de una región de tamaño $\ell$ alrededor del
  origen. Demuestra que el potencial vectorial $\bm A(\bm r)$ a una
  distancia $r\gg \ell$ está dada aproximadamente por
  $$
  \bm A(\bm r)=\frac{\bm m\times\bm r}{r^3},
  $$
  donde
  $$
  \bm m=\frac{1}{2c}\int d^3r\, \bm r\times \bm j(\bm r)
  $$
  es el {\em momento dipolar magnético} del sistema

  Pistas:
  \begin{itemize}
  \item Parte de la expresión
    $$ \bm A(\bm r)=\frac{1}{c}\int d^3r' \frac{\bm j(\bm r')}{|\bm
      r-\bm r'|}.
    $$
  \item Haz una expansión de Taylor para $\bm r'$ pequeña
    $$
    \frac{1}{|\bm r-\bm r'|}=\frac{1}{r}-\bm r'\cdot\nabla\frac{1}{r}+\ldots.
    $$
  \item El primer primer término lleva a
    $$\bm A^{(0)}(\bm r)=\frac{1}{rc}\int d^3r'\,\bm j(\bm r')=0,$$
    de acuerdo al problema anterior.
  \item Separa la contribución del segundo término en en un término simétrico y uno
    antisimétrico,
    $$A_i^{(1)}(\bm r)=\frac{r_j}{cr^3}\int d^3r'\, j_i(\bm r')r'_j=A_i^s+A_i^a,$$
    con
    $$A_i^s(\bm r)=\frac{r_j}{2cr^3}\int d^3r'\, (j_i(\bm
    r')r'_j+r'_i j_j(\bm r'))$$
    y
    $$A_i^a(\bm r)=\frac{r_j}{2cr^3}\int d^3r'\, (j_i(\bm
    r')r'_j-r'_i j_j(\bm r')).$$
  \item De acuerdo al problema anterior, la integral en el término
    simétrico es cero, $\bm A^s=0$.
  \item En el término antisimétrico identifica
    $\bm j(\bm r')\bm r\cdot\bm r'-\bm r'\bm r\cdot\bm j(\bm r')=\bm r\times(\bm
    j(\bm r')\times\bm r')$.
  \item Identifica en la integral resultante al momento dipolar
    magnético $\bm m$.
  \end{itemize}
  Notas:
  \begin{itemize}
  \item El término monopolar, proporcional a $1/r$ es nulo. No hay
    monopolos.
  \item El término dominante es el dipolar, proporcional a $1/r^2$, a
    menos que $\bm m=\bm 0$, en cuyo caso se deben explorar los
    siguientes términos en la expansión (cuadrupolar, octopolar,
    etc.).
  \item El potencial vectorial {\em rodea} al dipolo magnético de
    acuerdo a la regla de la mano derecha.
  \end{itemize}

\question Demuestra que el campo producido por un dipolo
  magnético es
  $$
  \bm B(\bm r)=\frac{3\bm m\cdot\bm r\bm r-\bm m r^2}{r^5}.
  $$
  Pista:
  \begin{itemize}
  \item Calcula el rotacional del potencial vectorial arriba.
  \end{itemize}
  Nota:
  \begin{itemize}
  \item El campo dipolar decae como $1/r^3$.
  \item Hay un fuerte parecido entre el campo dipolar magnético y el
    campo dipolar eléctrico, i.e., $\bm B$ es a $\bm m$ como $\bm E$ es
    a $\bm p$ lejos del dipolo.
  \end{itemize}

\question Considera una corriente estacionaria $I$ que circula en un circuito
  $\mathcal C$. Demuestra que el dipolo magnético es $\bm m=I\bm
  {\mathcal S}/c$, donde $\bm {\mathcal S}$ es el área del circuito,
  i.e. es un vector cuya componente $\mathcal S_x$ es el área de la
  proyección del circuito sobre el plano $yz$, con contribuciones
  positivas para recorridos de acuerdo a la regla de la mano derecha
  con el pulgar a lo largo de $x$, i.e., contrario a las manecillas
  del reloj visto desde $x>0$ hacia el origen, y contribuciones
  negativas para circuitos que se recorran en la dirección
  opuesta. Las otras componentes, $\mathcal S_y$ y $\mathcal S_z$ se
  definen análogamente.

  Pistas:
  \begin{itemize}
  \item El integrando que define $\bm m$ sólo es distinto de cero en
    el alambre que lleva la corriente.
  \item Considera un fragmento de alambre conductor, parte del
    circuito, con sección transversal $s$, longitud $dl$ y dirección
    $\hat{\bm t}$. Su contribución a la integral que define a $\bm m$
    se puede obtener reemplazando $d^3r$ por $sdl$.
  \item Reemplaza $sdl\bm j$ por $Id\bm l$, identificando $\bm
    j=j\hat{\bm t}$, $d\bm l=dl\hat{\bm t}$ e $I=js$.
  \item Identifica $d\bm a=\bm r\times d\bm l/2$ con el vector área de un triángulo
    angosto con un vértice en el origen y uno de cuyos lados es $d\bm
    l$.
  \item Suma sobre todos los elementos de longitud del circuito.
  \end{itemize}
  Nota:
  \begin{itemize}
  \item En un circuito plano, i.e., contenido en un plano,
    $\bm{\mathcal S}=\mathcal S\hat{\bm  n}$ donde $\mathcal S$ es el área
    de la región rodeada por el circuito y $\hat{\bm n}$ es un vector
    unitario perpendicular al plano, cuya dirección se establece con
    la regla de la mano derecha
  \end{itemize}


\question Considera un circuito circular de radio $a$ en el que
  circula una corriente $I$ y que descansa en el plano $xy$ centrado
  en el origen. Calcula el campo $\bm B(\bm r)$ a distancias $r\gg a$.

  Pista:
  \begin{itemize}
  \item Usa los resultados de los últimos problemas.
  \end{itemize}
  Nota:
  \begin{itemize}
  \item En la tarea anterior calculamos el resultado exacto a lo largo
    del eje $z$. Verifiquen que es compatible con el resultado de la
    aproximación dipolar.
  \end{itemize}

\question Considera una esfera de radio $a$ centrada en el origen
  uniformemente cargada en su interior con carga $Q$ y
  que gira con velocidad angular constante $\omega$ alrededor del eje
  $z$. Calcula el campo $\bm B(\bm r)$ que produce a distancias $r\gg
  a$.

  Pista:
  \begin{itemize}
  \item Calcula la densidad de carga, la velocidad con que se mueve
    cada elemento de volumen, su correspondiente densidad de corriente
    eléctrica, su contribución al momento dipolar magnético y el
    momento dipolar magnético total.
  \end{itemize}
  Nota:
  \begin{itemize}
  \item El campo dipolar magnético en este caso coincide con la
    solución exacta y vale en todo el exterior de la esfera.
  \end{itemize}

\question Considera un circuito en forma de cuadrado de lado $L$ que
  descanza en el plano $xy$ centrado en el origen y con los lados
  paralelos a los ejes cartesianos, en el que circula una corriente
  $I$ en dirección opuesta a las manecillas del reloj visto desde el
  eje $z$.
  Se aplica un campo $\bm B$ homogéneo en el plano $yz$ formando
  un ángulo $\theta$ con el eje $z$, i.e., $\bm B=B(0,-\sin\theta,
  \cos\theta)$. Demuestra que el circuito es sujeto a una torca
  $\tau=mB\sin\theta$ que apunta en la dirección $x$, i.e., $\bm
  \tau=\bm m\times\bm B$.

  Pistas:
  \begin{itemize}
  \item Calcula las fuerzas $\pm (IL/c)\hat{\bm x}\times \bm B$  y $\pm
    (IL/c)\hat{\bm y}\times \bm B$ sobre cada uno de los cuatro lados
    del circuito.
  \item Nota que la fuerza total es nula.
  \item Multiplica cada fuerza por su correspondiente brazo de
    palanca y suma para obtener la torca total.
  \end{itemize}


\question Considera un dipolo magnético $\bm m$ centrado en el origen
  sujeto a un campo externo $\bm B^e(\bm r)$ que varía lentamente en
  el espacio. Demuestra que sobre el dipolo actua
  \begin{parts}
    \part una fuerza $\bm F=\nabla (\bm m\cdot\bm B(\bm r))|_{\bm
        r=\bm 0}$ y
    \part una torca $\bm \tau=\bm m\times \bm B(\bm 0)$.
  \end{parts}

  Pistas:
  \begin{itemize}
  \item Considera la distribución de corriente $\bm j(\bm r)$. La
    fuerza sobre un elemento de volumen es $d\bm f=d^3r\,\bm j(\bm
    r)\times\bm B(\bm r)/c$.
  \item Haz una aproximación de Taylor del campo,
    $$
    \bm B(\bm r)=\bm B(\bm 0)+\bm r\cdot\nabla\bm B(\bm 0).
    $$
  \item Integra sobre el volumen en que circula la corriente
    $$
    \bm F=\frac{1}{c}\left(\int d^3r\, \bm j(\bm r)\right)\times \bm
    B(0)
    + \frac{1}{c}\left(\int d^3r\, \bm j(\bm r) \times (\bm
      r\cdot\nabla\bm B(0))\right).
    $$
  \item La integral en el primer término es cero, como en un problema
    previo.
  \item Para el segundo término podemos usar notación de índices,
    $$
    F_i=\frac{1}{c}\int d^3r\, \epsilon_{ijk} j_j(\bm r)r_l\partial_l
    B_k(0),
    $$
    con $\epsilon_{ijk}$ el tensor de Levi-Civita.
  \item El término $j_k r_l$ puede separarse en un término simétrico y
    uno antisimétrico.
  \item Como en un problema previo, el término simétrico integra a
    cero.
  \item El término antisimétrico se puede manipular de varias formas
    para obtener $\bm F=(\bm m\times\nabla)\times \bm B$. Por ejemplo
    \begin{eqnarray*}
      F_i=&\frac{1}{2c}\epsilon_{ijk}\int d^3r\,
            (j_jr_l-j_lr_j)\partial_l B_k(\bm 0)\\
      =&\frac{1}{2c}\epsilon_{ijk}\epsilon_{jlm}\epsilon_{nom}\int d^3r\,
         j_nr_o\
         \partial_l B_k(\bm 0)\\
      =&-\epsilon_{ijk}\epsilon_{jlm}m_m \partial_l B_k(\bm 0)\\
      =&\epsilon_{ijk} (\bm m \times \nabla)_j B_k(\bm 0)\\
      =&\left((\bm m \times \nabla) \times \bm B(\bm 0)\right)_i
    \end{eqnarray*}
  \item Finalmente, se pueden usar identidades y la ley de Gauss
    $$\bm F=(\bm m\times\nabla)\times \bm B=\nabla(\bm m\cdot\bm
    B)-\bm m\nabla\cdot\bm B=\nabla(\bm m\cdot\bm B).
    $$
  \item Para la torca integramos el elemento de torca
    $$
    d\bm\tau= \bm r\times d\bm f=d^3r\, \frac{1}{c}\bm r\times(\bm j\times\bm B(\bm 0))
    $$
    usando identidades vectoriales
    $$
    \bm \tau=\frac{1}{c}\int d^3r\, (\bm j(\bm r)\bm r\cdot\bm B(\bm 0)-\bm
    r\cdot\bm j(\bm r) \bm B(\bm 0)).
    $$
  \item El segundo término tiene la integral de $\bm r\cdot\bm j(\bm
    r)=\nabla\cdot (r^2\bm j(\bm r)/2)-r^2\nabla\cdot\bm j(\bm
    r)/2=\nabla\cdot (r^2\bm j(\bm r)/2)$, donde se usa la ecuación de
    continuidad, caso estacionario. Este término se puede integrar
    usando el teorema de Gauss y da cero.
  \item El primer término puede serpararse en una parte simétrica y
    una antisimétrica,
    $$
    \bm \tau=\frac{1}{2c}\int d^3r\, (\bm j(\bm r)\bm r\cdot\bm B(\bm
    0)+ \bm r\bm j(\bm r)\cdot\bm B(\bm 0)+\frac{1}{2c}\int d^3r\, (\bm j(\bm r)\bm r\cdot\bm B(\bm
    0)- \bm r\bm j(\bm r)\cdot\bm B(\bm 0).
    $$
  \item La parte simétrica integra a cero, como en el primer problema
    arriba.
  \item En la parte antisimétrica podemos identificar el momento
    dipolar magnético y llegar al resultado final.
  \end{itemize}
\question Muestra que los resultados de los últimos dos problemas son
  compatibles con una energía potencial $U=-\bm m\cdot \bm B$ para un
  dipolo magnético $\bm m$ en presencia de un campo $\bm B$.

  Pistas:
  \begin{itemize}
  \item La fuerza y la torca se pueden derivar de la energía potencial
    $U$ a partir de
    $$
    \bm F=-\nabla U,
    $$
    y
    $$
    \tau_{\hat{\bm n}}=\hat{\bm n}\cdot\bm \tau=-\partial U/\partial\theta,
    $$
    donde $\theta$ es un ángulo alrededor de un eje de giro que apunta
    a lo largo de un vector unitario $\hat{\bm n}$ medido de acuerdo a
    la regla de la mano derecha.
  \end{itemize}


\question Considera dos dipolos magnéticos $\bm m_1$ y $\bm m_2$
  separados una distancia $d$. Calcula la fuerza entre ellos si
  \begin{parts}
  \part Ambos apuntan en la dirección $z$ y están separados a lo largo
    de $z$.
  \part Ambos apuntan en la dirección $z$ y están separados a lo largo
    de $x$.
  \part Uno apuntan en la dirección $z$ y el otro en la dirección
    $-z$, y están separados a lo largo de $z$.
  \part Uno apuntan en la dirección $z$ y el otro en la dirección
    $-z$, y están separados a lo largo de $x$.
  \end{parts}


\question Considera un cilindro de radio $a$ y altura $L$
  uniformemente magnetizado a lo largo de su eje con magnetización
  $\bm M$. Demuestra que el dipolo magnético total del cilindro es
  igual al que produciría una corriente superficial $K=cM$ recorriendo
  las paredes del cilindro alrededor de la magnetización, siguiendo la
  regla de la mano derecha.

  Pistas:
  \begin{itemize}
  \item Se define la magnetización $\bm M$ como el momendo dipolar magnético
    por unidad de volumen.
  \item Calcula el momento dipolar magnético del cilindro $\bm m=\pi
    a^2 L \bm M$ multiplicando magnetización por volumen.
  \item Así como la densidad de corriente (volumétrica) es corriente
    que atraviesa una superficia en su dirección normal por unidad de
    área, la densidad de corriente superficial es corriente que
    atraviesa de un lado a otro de una línea en la superficie por
    unidad de longitud.
  \item Una corriente superficial $K$ sobre la pared del cilindro
    corresponde a una corriente total $I=KL$ circulando alrededor del
    cilindro.
  \item Dicha corriente produce un momento dipolar magnético
    $I\mathcal S/c$, donde $\mathcal S=\pi a^2$ es el área del
    circuito.
  \item Luego $I\mathcal S/c=m$, $KL\pi a^2/c=\pi a^2 LM$, $K=cM$.
  \end{itemize}


\question Considera un material magnetizado a lo largo de $y$ con una
  magnetización $\bm M(\bm r)=M_y(x)\hat{\bm y}$ que sólo depende de
  $x$. Considera una superficie immersa en el material en forma de
  rectángulo de lados $L_x$ y $L_y$ en el plano $xy$ y con aristas en
  $x_0$, $x_1=x_0+L_x$, $y_0$ y $y_1=y_0+L_y$ Calcula la
  corriente que atraviesa dicho rectángulo.

  Pistas:
  \begin{itemize}
  \item Usa un modelo simple de magnetización, en que cada átomo tiene
    un electrón (carga $-e$) que circula con periodo $T$ en una trayectoria que
    encierra un área $\mathcal S$ con normal $\hat{\bm y}$. Hay $n$ de estas órbitas por unidad
    de volumen. Luego, la magnetización es $M_y=-ne\mathcal
    S/cT$. Permitimos que $n$, $\mathcal S$ y/o $T$ dependan de $x$.
  \item Sólo contribuyen a la corriente aquellas átomos que
    intersecten a la superficie rectangular. Y de éstas, las órbitas
    que la intersecten dos veces (o cualquier número par de veces) no
    contribuyen, pues cada vez que el electrón atraviesa en una
    dirección, atraviesa en la dirección opuesta para cerrar su
    circuito.
  \item Entonces, sólo aquellas trayectorias que pasen en una
    dirección a través de la superficie y regresen por fuera de la
    superficie pueden contribuir a la corriente. Esto es, sólo las
    trayectorias que encierran a la línea en $x_0$ y a la línea
    $x_1$.
  \item El número de trayectorias que encierran la línea $x_0$ es
    igual al número de átomos en un volumen cilíndrico de sección
    transversal $\mathcal S(x_0)$ y de longitud $L_y$, i.e.,
    $n(x_0)L_y\mathcal S(x_0)$. Cada una de estas trayectorias
    contribuye a la corriente $z$ con un electrón cada tiempo $T$
    atravesando hacia abajo, por lo
    que su contribución a la corriente es $n(x_0)L_y\mathcal S(x_0)
    e/T=-cM_y(x_0)L_y$.
  \item Análogamente, la contribución a la corriente en $x_1$ es $-n(x_1)L_y\mathcal S(x_1)
    e/T=cM_y(x_1)L_y$.
  \item Sumando, obtenemos la corriente total
    $I=c(M_y(x_1)-M_y(x_0))L_y$.
  \item Si hubiera muchos tipos de cargas y de órbitas, estas
    contrinuirían tanto a $I$ como a $M_y$ por lo cual el resultado
    seguiría siendo válido.
  \end{itemize}
  Notas:
  \begin{itemize}
  \item Podemos escribir entonces $I=\int_{x_0}^{x_1}
    dx\int_{y_0}^{y_1} dy\, c(\partial/\partial x) M_y$.
  \item Podemos interpretar este resultado en términos de una densidad
    de corriente
    $j_z=c(\partial/\partial x)M_y$.
  \item Si además hubiera una magnetización apuntando en $x$ y que
    dependiera de $y$ podríamos sumar su contribución a la corriente y
    generalizar al resultado anterior a $j_z=c((\partial/\partial
    x)M_y-(\partial/\partial y)M_x$.
  \item Si la magnetización apuntara en $z$ no contribuiría a $j_z$.
  \item Si $M_x$ dependiera de $x$ o $z$ o $M_y$ dependiera de de $y$
    o $z$ tampoco contribuiría a $j_z$.
  \item Si repitiéramos el cálculo anterior para otras superficies
    obtendíamos la generalización
    $\bm j=c\nabla\times \bm M$.
  \end{itemize}

\question Considera un material magnetizado finito. Demuestra que
  sobre su frontera circula una corriente superficial $K=c\bm M\times
  \hat{\bm n}$.

  Pistas:
  \begin{itemize}
  \item Afuera del material la magnetización es nula, por lo que hay
    una fuerte variación espacial de la magnetización al atravesar la
    superficie.
  \item Considera la vecindad de cierto punto en la superficie.
  \item Sin perdida de generalidad, puedes transladar y orientar
    el sistema coordenado para que dicho punto quede en el origen
    y la normal a la superficie apunte en la dirección $z$. Cerca de
    ese punto $(\partial_z)\bm M$ es muy grande pues $\bm M$
    pasa de un valor finito en el interior a cero en el
    exterior. Toma el límite en que dicha variación se hace
    infinitamente abrupta pero en una región infinitamente delgada.
  \item Entonces, cerca de la superficie hay corrientes muy grandes
    $j_x=-c\partial_z M_y$ y $j_y=c\partial_z M_x$. Integrando desde
    un punto inmediatamente dentro hasta un punto inmediatamente afuera
    obtén la corriente superficial $K_x=cM_y$ y $K_y=-cM_x$ donde
    $\bm M$ es la magnetización dentro del material.
  \item Escribe el resultado anterior de manera vectorial $\bm
    K=c \bm M\times\hat{\bm z}$.
  \item Generaliza a cualquier punto de la superficie con cualquier
    orientación,  $\bm K=c \bm M\times\hat{\bm n}$, donde $\hat{\bm
      n}$ apunta hacia el exterior.
  \end{itemize}

\question Demuestra que en la interface entre dos materiales 1 y 2
  magnetizados circula una corriente $\bm K=c\hat{\bm n}\times(\bm
  M_2-\bm M_1)$ donde $\hat{\bm n}$ es un vector normal unitario que
  apunta del medio 1 al 2.

  Pista:
  \begin{itemize}
  \item Usa el problema anterior introduciendo una pequeña región
    vacía entre los materiales y toma el límite en que su ancho se anula.
  \end{itemize}


\question Demuestra la ley de Ampère {\em macroscópica}
  $$
  \nabla\times\bm H=\frac{4\pi}{c}\bm j^e,
  $$
  donde $\bm B=\bm H+4\pi M$ y $\bm j^e$ es la corriente externa.

  Pista:
  \begin{itemize}
  \item Escribe la corriente total $\bm j=\bm j^e+\bm j^i$ en la ecuación
    de Ampère {\em microscópica}, donde $\bm j^i=c\nabla\times \bm M$ es la corriente
    interna debido a la magnetización del material.
  \end{itemize}


\question Demuestra que en una superficie que separa dos regiones 1 y
  2 con normal $\hat{\bm n}$ que apunta de 1 hacia 2 se cumple que:
  \begin{parts}
  \part La componente normal de la densidad de flujo magnético $\bm B$
    es contínua, $\Delta B_\perp=0$, $B_\perp(2)=B_\perp(1)$.
  \part La componente tangencial del campo magnético $\bm H$ es
    continua en ausencia de corrientes superficiales externas, $\Delta
    \bm H_\parallel=0$, $\bm H_\parallel(2)=\bm H_\parallel(1)$.
  \part La componente tangencial del campo magnético $\bm H$ es
    discontínua en presencia de una corriente externa superficial,
    $\hat n\times (\bm H(2)-\bm H(1))=(4\pi/c)\bm K^e$.
  \end{parts}

  Pistas:
  \begin{itemize}
  \item Integra la ley de Gauss magnética en pequeños cilindros
    Gaussianos con una tapa en la región 2 y otra en la región 1 y toma el
    límite en que ambas tapas aprisionan la superficie.
  \item Integra la ley de Ampère macroscópica en pequeños rectángulos
    con un lado en el medio 1 y otro en el medio 2 y toma el límite en
    que dos lados aprisionan la superficie.
  \item Otra alternativa es usar las condiciones de continuidad para
    $\bm B_\parallel$ y escribir la corriente superficial total en
    términos de la externa y de la discontinuidad de la magnetización.
  \end{itemize}

\question Una bobina está formada por $N\gg 1$ espiras de alambre
  enrolladas uniformemente alrededor de un cilindro de altura $\ell$ y
  radio $a\ll \ell$ lleno de un material con permeabilidad magnética
  $\mu$. Ignorando efectos de borde, halla los campos $\bm H$ y $\bm
  B$ en todo el espacio.

  Pistas:
  \begin{itemize}
  \item El campo magnético $\bm H$ y la densidad de flujo magnético
    $\bm B$ están ligados en muchos materiales a través de una
    simple relación lineal $\bm B=\mu\bm H$, donde $\mu$ se conoce
    como permeabilidad magnética y es un atributo de cada material.
  \item Usa argumentos de simetría y la ley de Gauss magnética para
    establecer que la dirección de los campos es axial.
  \item Integra la ley de Ampère macroscópica en rectángulos con dos
    lados axiales y dos lados radiales para demostrar que el campo es
    constante en el exterior y constante en el interior y para hallar
    la diferencia entre $H$ dentro y fuera de la bobina.
  \item Usa condiciones de contorno lejos de la bobina y obtén $H$ en
    todo el espacio.
  \item Obtén $B$ en todo el espacio.
  \end{itemize}

\question Considera un material superconductor, un diamagneto
  perfecto, llenando la región $z<0$ con vacío en la región $z>0$, y
  un dipolo magnético $\bm m$ que apunta en la dirección $z$ y
  colocado sobre el eje $z$ a una distancia $d$ del origen.
  \begin{parts}
  \part Demuestra que en $z>0$ el campo $\bm B$ es igual al campo
    producido en el espacio vacío por el dipolo $\bm m$ en $z=d$ mas el campo que produciría un
    dipolo ficticio $\bm m'=-\bm m$ colocado en $z=-d$.
  \part Halla la fuerza que siente el dipolo real. ¿Se ve atraido o
    repelido por la superficie?
  \part ¿Cambiaría el resultado si rotamos el dipolo 180 grados?
  \end{parts}

  Pistas:
  \begin{parts}
  \part Un diamagneto es un material con $\mu<1$.
  \part Un diamagneto {\em perfecto}, como lo son los
    superconductores, tiene $\mu=0$.
  \part Calcula $\bm B$ en el sistema real, en $z<0$.
  \part Aplica condiciones de contorno para hallar $B_z(0^+)$.
  \part Demuestra que estas se cumplen por la solución propuesta.
  \part ¿Cuánto vale la densidad de corriente {\em externa} en la
    superficie del superconductor?
  \part Demuestra que la solución propuesta también cumple las condiciones de contorno
    para $\bm H_\parallel$ en $z=0$.
  \part La fuerza sobre un dipolo en términos del campo fue obtenida
    en un problema previo.
  \end{parts}

\question Demuestra que en una región simplemente conexa en que no hay corrientes
  eléctricas externas y hecha de un material homogéneo el campo
  magnético se puede obtener de un potencial {\em escalar} $\phi_m$,  $\bm
  H=-\nabla\phi_m$,y que satisface la ecuación de Laplace
  $\nabla^2\phi_m=0$.

  Pistas:
  \begin{parts}
  \part En dicha región se cumplen las ecuaciones $\nabla\cdot\bm B=0$ y
    $\nabla\times\bm H=0$.
  \part Usa la segunda para escribir $\bm H=-\nabla\phi_m$ en términos
    de algún potencial escalar $\phi_m$.
  \part Usa la relación constitutiva $\bm B=\mu \bm H$ y escribe la
    primera como $-\nabla\cdot\mu\nabla\phi_m=0$.
  \end{parts}
  Nota:
  \begin{itemize}
  \item Si la región no es simplemente conexa el potencial escalar
    podría resultar multivaluado. Hay que tener cuidado con estos casos.
  \item Un ejemplo sería el de un alambre recto infinitamente largo en
    que circula una corriente $I$. El campo $\bm B$ es tangencial,
    $B_\theta=2I/rc$, y puede derivarse de un potencial escalar
    $\phi_m(\bm r)=\phi_m(\theta)=-2I\theta/c$, pero este potencial
    cambia de valor al dar una vuelta completa, i.e.,
    $\phi_m(\theta+2\pi)\ne\phi_m(\theta)$.
  \end{itemize}


\question Considera una esfera aislante no magnetizable de radio $a$ centrada en el
  origen en cuya superficie se deposita una carga con densidad
  uniforme $\sigma$. La esfera se pone a girar alrededor del eje $z$
  con velocidad angular $\omega$. Calcula el campo magnético producido
  en todo el espacio.

  Pistas;
  \begin{itemize}
  \item En el interior de la esfera el campo se puede describir por un
    potencial escalar $\phi_<$ que obedece la ecuación de Laplace.
  \item Lo mismo en el exterior, con otro potencial escalar $\phi_>$.
  \item El sistema tiene un dipolo magnético no nulo.
  \item Calcula la corriente superficial debida a la rotación de la
    carga superficial en términos de $\sigma$ y $\omega$ en función de
    $\theta$. Verifica que es proporcional a $\sin\theta$.
  \item Prueba entonces como anzatz potenciales que van como
    $\cos\theta$ y que satisfagan la ecuación de Laplace.
  \item En el exterior prueba un potencial dipolar que decaiga en
    $r\to\infty$: $\phi_>(r,\theta)=\alpha\cos\theta/r^3$.
  \item En el interior prueba que no sea singular en el origen,
    $\phi_<(r,\theta)=\beta r\cos\theta$.
  \item Calcula $\bm B$ en el interior y el exterior.
  \item Aplica condiciones de contorno sobre $B_\perp$ y $\bm
    B_\parallel$ y determina las constantes $\alpha$ y $\beta$.
  \item Calcula el momento dipolar magnético $\bm m$ de la esfera
    rotante. Relacionalo con las constantes $\alpha$ y $\beta$.
  \end{itemize}
  Nota:
  \begin{itemize}
  \item Nota que el potencial no está definido en la región
    superficial, donde sí hay corriente, y que no hay continuidad
    entre sus valores fuera y dentro de la esfera ni hay por qué esperarla.
  \end{itemize}



\end{questions}
\end{document}
